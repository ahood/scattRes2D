\section{Zeros of Hankel functions}
\label{sec-hankel_zeros}

The DtN map boundary condition involves things that look like
$k\left(H_n^{(1)}\right)'(kR)/H_n^{(1)}(kR)$, for which we need
rational approximations. Therefore we need to know the locations
of the zeros of $\H$. 

\subsection{Zeros from McMahon expansion}

In Section 3 of Boro D\"oring's paper,
it's stated that large zeros of $\H$ satisfy
\[
 z_s \sim \beta - \sum_{k=0}^\infty \frac{D_{2k+1}}{\beta^{2k+1}},
 \qquad
 \beta = \left(s + \frac{n}{2} - \frac14\right)\pi - \alpha,
 \qquad
 \alpha = \frac{i}{2}\ln 2
\]
where $s$ is an integer index (which I believe should start 
from 1).

The offensively-indexed algorithm given therein for computing
$D_{2k+1}$ is not reproduced here lest I make a mistake.
I came up with my own version (changed indexing, switched a
few steps around) that was easier for me to code. 
For $\nu = 1,2,...$
\begin{enumerate}
 \item $t_\nu = -\left(\nu - \frac12\right)
                \frac{n^2 - \left(\nu-\frac12\right)^2}{\nu}
                \cdot t_{\nu-1}; \quad t_0 = 1$
 \item $A_\nu = -\displaystyle\sum_{\lambda=0}^{\nu-1}
                (-1)^{\nu-\lambda} A_\lambda
                t_{\nu-\lambda}; \quad A_0 = 1$
 \item $Q_\nu = \displaystyle\sum_{\lambda=0}^{\nu}
                C_{\nu-\lambda}C_\lambda; \quad C_0 = 1$
 \item $F_\nu^{(\mu)} = F_{\nu-1}^{(\mu-1)} - 
                        \displaystyle\sum_{\lambda=1}^{\nu-\mu}
                        Q_\lambda F_{\nu-\lambda}^{(\mu)},
                        \quad \mu = 1,2,...,\nu;
                        \quad F_{\nu-1}^{(0)} = C_{\nu-1}$
 \item $C_\nu = -\frac{A_\nu}{2\nu-1}; \quad C_0 = 1$
 \item $D_\nu = C_\nu + \displaystyle\sum_{\lambda=1}^{\nu-1}
                        D_\lambda F_\nu^{(\lambda)}$.
\end{enumerate}
The way that I coded this algorithm, I included the index 0
terms in some of the vectors and not in others; there 
isn't much confusion because I vectorized the sums
involved. Just for posterity, with the new indexing
\[
 z_s \sim \beta - \sum_{\nu=1}^\infty
 \frac{D_\nu}{\beta^{2\nu - 1}}.
\]

\subsection{Zeros from Olver expansion}

In Section 4 of Boro D\"oring's paper, 
approximate zeros of $\H$ are of the form
$nt(a_se^{-2\pi i/3})$ and $-nt(b_se^{2\pi i/3})$, where
$t$ is defined by
\[
 t(x) = \frac{1}{\cosh\sigma(x)}, \qquad
 \sigma(x) - \tanh\sigma(x) = \frac23 \cdot \frac1n \cdot
 x^{3/2} \equiv \rho(x)
\]
and $a_s$ and $b_s$ are the negative zeros of the
first and second kind Airy functions, respectively.

In order to compute them, I used the following 
approximations from Wikipedia:
\[
 Ai(-x) 
 \sim 
 \frac{\sin\left(\frac23 x^{3/2} + \frac{\pi}{4}\right)}
      {\sqrt{\pi} x^{1/4}},
 \qquad
 Bi(-x) 
 \sim 
 \frac{\cos\left(\frac23 x^{3/2} + \frac{\pi}{4}\right)}
      {\sqrt{\pi} x^{1/4}}.
\]
With these,
\[
 a_s \approx -\left[ \frac32 \pi \left(k-\frac14\right)
              \right]^{2/3}, 
 \quad k > 0
\]
and
\[
 b_s \approx -\left[ \frac32 \pi \left(k+\frac14\right)
              \right]^{2/3}, 
 \quad k \ge 0.
\]
The values of $k$ are determined by needing $a_s$ and
$b_s$ to be negative.

