\section{Introduction}
\label{sec-intro}

The code is for quantum scattering and resonance computation
for a class compactly-supported potentials.
In addition to being supported
in some disc $B(0,R)$, the potential $V$ must be smooth except possibly
on a finite number of circles $r = r_0$. For 
incident wave $\inc(r,\theta) = e^{ikr\cos\theta}$ with frequency
parameter $k$, the corresponding scattered wave obeys the equation
\begin{equation}\label{eqtn-scattprob}
\begin{aligned}
 \left(-\Delta + V - k^2\right)\scatt = -V\inc
 \qquad
 \text{on }B(0,R), \\ 
 \text{$\scatt$ outgoing}
 \qquad
 \text{on }\partial B(0,R).
\end{aligned}
\end{equation}
The boundary condition is enforced exactly with the Dirichlet-to-Neumann
(DtN)
map, approximately with a PML, and Dirichlet boundary conditions
work pretty well under special circumstances. Therefore, the user
can choose any of the three in order to compute $\scatt$.
The energy of the scattered wave is $E = k^2$.

The resonances are defined as the energies $E$ such that
\begin{equation}\label{eqtn-resprob}
\begin{aligned}
 \left(-\Delta + V - k^2\right)\scatt = 0
 \qquad
 \text{on }B(0,R), \\
 \text{$\scatt$ outgoing on }\partial B(0,R)
\end{aligned}
\end{equation}
has a nonzero solution. The user can still use Dirichlet boundary conditions
or a PML to enforce that the scattered wave is outgoing. 
It should be noted that we won't use the
DtN map for resonance computation, because the 
resulting eigenvalue problem is nonlinear (and non-polynomial and
non-rational). The user can choose to use a {\it rational approximation}
to the DtN map instead (which leads to a rational eigenvalue problem).


