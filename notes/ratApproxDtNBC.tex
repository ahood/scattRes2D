\section{{\tt ratApproxDtNBC}}
\label{sec-ratApproxDtNBC}

We've finally arrived at the object that will let us
compute resonances using the DtN map boundary condition.
Recall that what we
really care about is $E = k^2$ (resonance energies). So, the
user should be specifying a region in which to look for
$E$'s, not $k$'s. But rewriting $\dtnT(k)$ in terms of $E$ introduces
a square root. So, the user needs to specify which branch of
the square root to use.
\begin{equation}
\begin{aligned}
 \text{first  branch} &\qquad z &\mapsto  \sqrt{z} \\
 \text{second branch} &\qquad z &\mapsto -\sqrt{z}
\end{aligned}
\end{equation}
Having chosen a mapping $E \mapsto k$, the code will 
create matrices $A$ and $B$ such that
$\ratT_{\text{full}}(k) := \ratA - k^2 \ratB$ has Schur complement
\begin{equation}
 \ratT(k) = \dtnA - k^2 \dtnB + \ratC(k),
\end{equation}
where $\ratC(k) \approx \dtnC(k)$.

\subsection{Creating a {\tt ratApproxDtNBC} instance}

We'll obviously need everything we needed to make a
{\tt DtNBC} instance, so the easiest thing is to make
and pass a {\tt DtNBC} instance (call it {\tt dtn}). 
The user should also choose a region in which to look for
resonance energies $E$ and a branch of the square root
{\tt br}. Then make a {\tt ratApproxDtNBC} instance with
\begin{verbatim}
 >> rat = ratApproxDtNBC(dtn,ell,br);
\end{verbatim}

\subsection{Properties and methods}

\begin{description}
 \item[Properties]
   \begin{itemize}
    \item[]
    \item {\tt A,B} are the matrices $\ratA$ and $\ratB$
    \item {\tt fourierA12, fourierA21} are the matrices $\tilde{A}_{12}$ and
          $\tilde{A}_{21}$ in Section~\ref{sec-preconditioners}, but are not
          important to the user
    \item {\tt Aschur} is the matrix whose eigenvalues are the same as
          the finite eigenvalues of the $(\ratA,\ratB)$ pencil (obtained by
          Schur complementing away the zero diagonal entries of $\ratB$)
    \item {\tt br} is the branch of the square root specified by the user
    \item {\tt dtn} is the {\tt DtNBC} instance passed by the user
    \item {\tt ell} is the {\tt ellipse} instance passed by the user 
    \item {\tt ratf} is a vector of {\tt ratApproxWithPoles}
          instances, one for each component of the DtN map
    \item {\tt poles} is a cell array holding the poles of each
          component of the DtN map within the specified ellipse
    \item {\tt sc} is a Schur complement object keeping track of how
          to Schur complement $A-k^2 B$ into an approximation to $\dtnT(k)$
   \end{itemize}
 \item[Methods]
   \begin{itemize}
    \item[]
    \item {\tt resonances(obj,n,E0,verbose)} returns the {\tt n}
          resonance approximations closest to {\tt E0} and their
          null vectors as the pair {\tt resvecs, resvals}
    \item {\tt compute\_error(obj,r)} see {\tt ratApprox}
    \item {\tt show\_error(obj,r)} ditto
    \item {\tt show\_poles(obj,r)} ditto
    \item {\tt T(obj,k)} returns the matrix $\ratT(k)$ described above
    \item {\tt Tfull(obj,k)} returns the matrix $\ratT_{\text{full}}(k)$
          described above
   \end{itemize}
\end{description}

\subsection{Computing resonances}

With an instance {\tt rat} created, 
finding the $n$ approximate resonances
close to the center of the ellipse 
is done with
\begin{verbatim}
 >> [resvecs,resvals] = rat.resonances(n,ell.c);
\end{verbatim}
As you go farther from the center of the ellipse the
approximations will first get less accurate and then 
become spurious.
