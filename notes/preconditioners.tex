\section{Preconditioners}
\label{sec-preconditioners}

For the problem posed with DtN boundary conditions,
we're looking at the matrix $\dtnT(k) = \dtnA - k^2 \dtnB +\dtnC(k)$ where
$\dtnC(k) = -(U \otimes I)(\operatorname{diag}(d) \otimes e_Ne_N^T)(U^{-1} \otimes I)$,
$\dtnA$ comes from discretizing the Laplacian and $\dtnB$ is the
identity, but with certain rows of $\dtnA$ and $\dtnB$ set to
enforce the derivative part of
$\psi_r(R,\theta) - f(k,R)\psi(R,\theta) = 0$ (where
$f(k,R)$ is the DtN map described in Section~\ref{sec-dtn_map}).

The diagonal and block diagonal of $\dtnT(k)$ are obvious preconditioners 
to use for solving the scattering problem~(\ref{eqtn-defT})
iteratively,
and they do significantly improve the time it takes. Another
thing we can do is take advantage of the fact that the Fourier transform
diagonalizes the differential operator and the DtN map. 

\subsection{Action of $\dtnT(k)$ in Fourier space}

We already know
that $\operatorname{diag}(d) \otimes e_Ne_N^T$ acts on vectors of the form
$[\hat{\psi}_{-N+1}(\hat{r}), \hdots , \psi_N(\hat{r})]^T$ (i.e. the
vector of Fourier coefficients evaluated at the radial mesh
points; see Section~\ref{sec-dtnC}), so $(U \otimes I)$ maps
$[\psi_{-N+1}(\hat{r}), \hdots , \psi_N(\hat{r})]^T$ to
$[\psi(\hat{r},\theta_1), \hdots , \psi(\hat{r},\theta_{2N})]^T$
and so all we need to do to get the action of
$\dtnT(k)$ in Fourier space is conjugate by
$U \otimes I$.

Due to the fact that certain rows of $\dtnA$ and $\dtnB$ have been
messed up by imposing boundary conditions, we'll get a nice
(mostly) block preconditioner only if we ignore those changes.
Let's call the matrices without the boundary conditions imposed
$\widehat{\dtnA}$ and $\widehat{\dtnB}$. Then
$\widehat{\dtnA} = -L + \operatorname{diag}(V(\hat{r},\theta_1), \hdots ,
V(\hat{r},\theta_N))$ and $\widehat{\dtnB} = I$, where
\begin{align*}
L &= L_r + D_{\theta\theta} \otimes R^2 \\
  &= I_{2N} \otimes \left({\tt Drr\_I} + R {\tt Dr\_I} \right) +
     \underbrace{
     \begin{bmatrix} 0 & I_N \\ I_N & 0\end{bmatrix}
     }_{J_{2N}}
     \otimes \left({\tt Drr\_J} + R {\tt Dr\_J} \right) +
     D_{\theta\theta} \otimes R^2
\end{align*}
and $R = \operatorname{1./\hat{r}}$.
Conjugating $L$ then yields
\begin{align*}
(U^{-1} \otimes I)L(U \otimes I) =
         I_{2N}   \otimes \left({\tt Drr\_I} + R {\tt Dr\_I} \right) +
  U^{-1} J_{2N} U \otimes \left({\tt Drr\_J} + R {\tt Dr\_J} \right) +
  U^{-1} D_{\theta\theta} U \otimes R^2.
\end{align*}
Altogether we get
\begin{align*}
(U^{-1} \otimes I)
(\widehat{\dtnA} &- k^2 \widehat{\dtnB} + \dtnC(k))
(U \otimes I) = \\ 
  & -         I_{2N}    \otimes \left({\tt Drr\_I} + R {\tt Dr\_I} \right) \\
  & - (U^{-1} J_{2N} U) \otimes \left({\tt Drr\_J} + R {\tt Dr\_J} \right) \\
  & - (U^{-1} D_{\theta\theta} U) \otimes R^2 \\
  & + (U^{-1} \otimes I)\operatorname{diag}(V)(U \otimes I)
  - k^2 I - \diag\left(d(k,R)\right) \otimes e_n e_n^T.
\end{align*}
By experiment,
$U^{-1} J_{2N} U = \operatorname{diag}( (-1)^{N_s})$ and
$U^{-1} D_{\theta\theta} U = \operatorname{diag}( -N_s^2 )$,
where $N_s = [-N + 1 , \hdots , 0, 1, \hdots , N]$.
The second one makes sense because
\begin{align*}
 \Delta\psi
 = \sum_n \left(\psi_n''(r) + \frac{1}{r}\psi_n'(r)
                            - \frac{n^2}{r^2}\psi_n(r)\right)e^{in\theta}
\end{align*}
and the $-n^2/r^2$ term corresponds to
$\operatorname{diag}(-N_s^2) \otimes R^2$. I don't have a similar explanation
for the first one and {\bf I haven't verified it on paper}.

Anyway, using these relationships,
\begin{align*}
(U^{-1} \otimes I)
(\widehat{\dtnA} &- k^2 \widehat{\dtnB} + \dtnC(k))
(U \otimes I) = \\ 
 &-I_{2N} \otimes \left({\tt Drr\_I} + R {\tt Dr\_I} \right) \\
 &-\diag( (-1)^{N_s} )
          \otimes \left({\tt Drr\_J} + R {\tt Dr\_J} \right) \\
 &-\diag( -N_s^2 )     \otimes R^2 \\
 &+ (U^{-1} \otimes I)\diag(V)(U \otimes I)
  - k^2 I - \diag\left(d(k,R)\right) \otimes e_n e_n^T.
\end{align*}
We can approximate the $V$ term by a block diagonal thing in various ways
(dropping off-diagonal part, or approximating the potential by
something axisymmetric), but either way let the $j$-th block of
the approximation be $\hat{V}_j$. Then the $j$-th block  of
$(U^{-1} \otimes I)
(\widehat{\dtnA} - k^2 \widehat{\dtnB} + \dtnC(k))(U \otimes I)$ is
\begin{align*}
 \hat{X}_j(k) &=
 -                  \left({\tt Drr\_I} + R {\tt Dr\_I}\right)
 -(-1)^{-N + j} \left({\tt Drr\_J} + R {\tt Dr\_J}\right) \\ 
 &+(-N + j)^2 R^2 + \hat{V}_j - k^2 I - d(k,R)_j e_n e_n^T
\qquad
(j = 1,2,...,2N).
\end{align*}
Therefore
\begin{align*}
 \widehat{\dtnA} - k^2 \widehat{\dtnB} + \dtnC(k) \approx
 (U \otimes I)\operatorname{blkdiag}(\hat{X}_1(k), \hdots , \hat{X}_N(k))
 (U^{-1} \otimes I).
\end{align*}
However, without enforcing boundary conditions this matrix is not invertible,
which is expected because enforcing boundary conditions is how we get
a unique solution. So, now we have to consider boundary conditions.

One option, simple-minded as it seems, is to take the matrix
\begin{align*}
 \hat{F}(k) 
 = 
 (U^{-1} \otimes I)
 \left[(\dtnA - \widehat{\dtnA}) - k^2(\dtnB - \widehat{\dtnB})\right]
 (U      \otimes I),
\end{align*}
which satisfies
\begin{align*}
 \left(\dtnA - k^2 \dtnB + \dtnC(k)\right)
 = 
 \left(\widehat{\dtnA} - k^2 \widehat{\dtnB} + \dtnC(k)\right)
 + (U \otimes I)\hat{F}(k)(U^{-1} \otimes I),
\end{align*}

\subsection{Fourier preconditioner for $\dtnT(k)$}

Take the block diagonal part of $\hat{F}(k)$ to make the preconditioner
\begin{align*}
 M(k) 
 :&= 
 (U \otimes I)
 \operatorname{blkdiag}(\hat{X}_1(k),...,\hat{X}_N(k))
 (U^{-1} \otimes I)
 + 
 (U \otimes I)
 \operatorname{blkdiag}(\hat{F}_1(k),...,\hat{F}_N(k))
 (U^{-1} \otimes I) \\ 
 &= 
 (U \otimes I)\hat{M}(k)(U^{-1} \otimes I) \\ 
 &= 
 (U \otimes I)
 \operatorname{blkdiag}(\hat{M}_1(k),...,\hat{M}_N(k))
 (U^{-1} \otimes I)
 \qquad
 \left(\hat{M}_j(k) = \hat{X}_j(k) + \hat{F}_j(k)\right)
\end{align*}

Since the $j$-th block row
of $U^{-1} \otimes I$ is $U^{-1}(j,:) \otimes I$ and similarly for
$U \times I$,
\begin{align*}
 \hat{F}_j(k)
 = 
 (U^{-1}(j,:) \otimes I)
 \left[(\dtnA - \widehat{\dtnA}) - k^2(\dtnB - \widehat{\dtnB})\right]
 (U     (:,j) \otimes I).
\end{align*}
We save space by realizing that 
$(\dtnA - \widehat{\dtnA}) - k^2(\dtnB - \widehat{\dtnB})$
is very sparse and we can write it as an outer product. Details
omitted here, but this has been implemented and tested.

%% We can enforce boundary conditions in Fourier space while still taking
%% advantage of the block diagonal nature of the operator, though.
%% As described in the DtN map section, if we write $\psi_{scatt} =    
%% \sum c_n(r) e^{in\theta}$, then the boundary condition becomes       
%% $c_n'(R) = f_n(k,R)c_n(R)$.   

\subsection{Fourier preconditioner for $\ratT(k)$ and $\ratT_{\text{full}}(k)$}

The Fourier preconditioner for $\ratT(k)$ is constructed the same way except
using a rational approximation to $d(k,R)$. 
Now,
\begin{align*}
 \ratT_{\text{full}}(k) 
 = 
 \begin{bmatrix} \dtnA - k^2 \dtnB & A_{12} \\
                 A_{21}            & D_{22} - k^2 I
 \end{bmatrix},
\end{align*}
which is pretty similar to the variable names chosen in the code.
To reiterate, this means
\begin{align*}
 \ratT(k) 
 =
 \dtnA - k^2 \dtnB 
 \underbrace{
 - A_{12}(D_{22} - k^2 I)^{-1} A_{21}
 }_{\ratC(k)}.
\end{align*}
It happens that $A_{12} = (U \otimes I) \tilde{A}_{12}$ and
$A_{21} = \tilde{A}_{21}(U^{-1} \otimes I)$ 
where the tilde matrices have very nice structure and will be very
fast to apply (see Section~\ref{sec-rat_approx}). For convenience, let
$\dtnA - k^2 \dtnB = (U \otimes I)\hat{Y}(k) (U^{-1} \otimes I)$,
where $\hat{Y}$ is the block diagonal matrix with blocks
$\hat{Y}_j(k) = \hat{M}_j(k) + \diag\left(d(k,R)\right)\otimes e_ne_n^T$.
Then
\begin{align*}
 \ratT_{\text{full}}(k) =
  \begin{bmatrix}
   (U \otimes I)\hat{Y}(k)(U^{-1}\otimes I) &
   (U \otimes I) \tilde{A}_{12} \\
   \tilde{A}_{21} (U^{-1} \otimes I) &
   D_{22} - k^2 I
  \end{bmatrix} =
  \begin{bmatrix}U \otimes I & 0 \\ 0 & I\end{bmatrix}
  \begin{bmatrix}
   \hat{Y}(k) & \tilde{A}_{12} \\
   \tilde{A}_{21} & D_{22} - k^2 I
  \end{bmatrix}
  \begin{bmatrix}U^{-1} \otimes I & 0 \\ 0 & I\end{bmatrix}.
\end{align*}

Since $D_{22}$ is diagonal, we could just drop $\tilde{A}_{12}$ 
and $\tilde{A}_{21}$
to get a block diagonal preconditioner, but that seems wasteful.
Consider the block 2x2 matrix inversion formula
\begin{align*}
 \begin{bmatrix} A & B \\ C & D\end{bmatrix}^{-1}
 =
 \begin{bmatrix}
  S^{-1} & -S^{-1}BD^{-1} \\
  -D^{-1} C S^{-1} & D^{-1} + D^{-1} C S^{-1} B D^{-1}
 \end{bmatrix}
\end{align*}
where $S$ is the Schur complement $A - BD^{-1}C$.
 Using this,
\begin{align*}
 &\begin{bmatrix}
  \hat{Y}(k) & \tilde{A}_{12} \\ 
  \tilde{A}_{21} & D_{22} - k^2 I\end{bmatrix}^{-1} \\ 
 &=
 \begin{bmatrix}
  S^{-1} & -S^{-1}\tilde{A}_{12}(D_{22} - k^2 I)^{-1} \\
  -(D_{22} - k^2 I)^{-1} \tilde{A}_{21} S^{-1} &
  (D_{22} - k^2 I)^{-1} +
  (D_{22} - k^2 I)^{-1} \tilde{A}_{21} S^{-1} \tilde{A}_{12}
  (D_{22} - k^2 I)^{-1}
 \end{bmatrix}.
\end{align*}
Note that here
\begin{align*}
 S = \hat{Y}(k) - \tilde{A}_{12} (D_{22} - k^2I)^{-1}
      \tilde{A}_{21}
   = (U^{-1} \otimes I)\ratT(k)(U \otimes I).
\end{align*}
Inverting $S$ can be done with the preconditioner we used for
$\ratT(k)$.

