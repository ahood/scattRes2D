\subsection{{\tt scattResComp2d\_sparse}}
\label{sec-scattResComp2d_sparse}

The point of this object is to keep track of all the same
things as {\tt scattResComp2d}, except instead of storing 
large matrices, we have
routines to apply them quickly (at least the important ones).
For the most part, using the two is exactly the same. In particular,
creation is the same:
\begin{equation}
\begin{aligned}
 &{\tt >> s = scattResComp2d(Nt,Nrs,Rs);} \\ 
 &{\tt >> s\_sparse = scattResComp2d\_sparse(Nt,Nrs,Rs);}
\end{aligned}
\end{equation}

Here are all the differences.
\begin{center}
\begin{longtable}{p{0.33\linewidth} p{0.33\linewidth} p{0.33\linewidth}}
{\bf Removed}          & {\bf Added}             & {\bf Why} \\ 
\hline \\ 
% ---------------------------------------------------
{\tt ICBCrows}         & {\tt localICBCrows}, 
                         {\tt globalICBCrows}    & \\ \hline \\
% ---------------------------------------------------
                       & {\tt apply\_U()}, 
                         {\tt apply\_U\_kron\_I()},
                         {\tt apply\_Uinv()},
                         {\tt apply\_Uinv\_kron\_I()} 
                       & For applying {\tt U} and {\tt Uinv}
                         quickly. \\ \hline \\ 
% ---------------------------------------------------
{\tt Dr},
{\tt Drr}              & {\tt E4}, {\tt D4},
                         {\tt E3t}, {\tt D3t}
                       & Avoid storing huge matrices
                         {\tt Dr} and {\tt Drr}, keep
                         some pieces for convenience. \\ \hline \\ 
% ---------------------------------------------------
{\tt Drs}              &                         
                       & Who cares. \\ \hline \\ 
% ---------------------------------------------------
{\tt It},
{\tt Jt}               & {\tt apply\_Drr\_J()},
                         {\tt apply\_J\_kron\_Drr\_J()},
                         {\tt apply\_Dr\_J()},
                         {\tt apply\_J\_kron\_Dr\_J()},
                         {\tt apply\_Dr\_I()},
                         {\tt apply\_I\_kron\_Dr\_I()},
                         {\tt apply\_Drr\_I()},
                         {\tt apply\_I\_kron\_Drr\_I()} 
                       & Don't store the highly structured
                         and sparse matrices, just apply where needed. \\ \hline \\ 
% ---------------------------------------------------
{\tt enforceDE()},
{\tt enforceIC()},
{\tt enforceBlockIC()} & {\tt apply\_op\_no\_bcs()} 
                       & The removed methods construct the
                         huge matrices $A$ and $B$ (with the
                         exception of the boundary condition rows,
                         which are set later)
                         before
                         applying $A - k^2 B$. Just apply
                         $A - k^2 B$ (with wrong boundary condition rows,
                         to be corrected later)
                         instead. \\ \hline \\ 
% ---------------------------------------------------
{\tt solve(...)}   & {\tt solve(...,precond,verbose)}
                       & Old solve directly inverts $T(k)$
                         and new one inverts $T(k)$ iteratively.
                         Has the option to pass a 
                         preconditioner (will use a default if none
                         passed) and to show GMRES output if verbose
                         set to true (defaults to false). \\ \hline \\ 
% ---------------------------------------------------
                       & {\tt ICBCmatrices()} 
                       & Sets up {\tt ArowChange}, 
                         {\tt BrowChange} without
                         creating $A$ and $B$. \\ \hline \\ 
% ---------------------------------------------------
                       & {\tt apply\_DtNmap()} 
                       & Applies $\dtnC(k)$ from DtN map
                         boundary condition formulation. \\ \hline \\ 
% ---------------------------------------------------
                       & {\tt apply\_kron(X,Y,xhat)} 
                       & Apply kronecker product
                         fast using rule $(B^T \otimes A)
                         \operatorname{vec}(X) = 
                         \operatorname{vec}(AXB)$. \\ 
\end{longtable}
\end{center}
