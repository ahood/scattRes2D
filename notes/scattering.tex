\section{Solving the scattering problem}

To solve~(\ref{eqtn-scattprob}), this code uses a
spectral collocation approach. This means we approximate the
values of the scattered wave on a Chebyshev-type mesh.
Different
boundary conditions correspond to different ways of
approximating the outgoing boundary condition. 
In the code, the discretization of the
PDE in~(\ref{eqtn-scattprob}) with 
each boundary condition corresponds
to its own object:
\begin{align*}
 \text{Dirichlet boundary condition on boundary of
  potential support}\quad\leftrightarrow\quad{\tt dirBC} \\
 \text{PML plus Dirichlet boundary condition at outer edge}
  \quad\leftrightarrow\quad{\tt pmlBC} \\
 \text{DtN map boundary condition on boundary of potential
  support}\quad\leftrightarrow\quad{\tt DtNBC}
\end{align*}
These three objects share many properties and methods 
(e.g. for defining discretizations, methods for plotting, etc.), 
so they are children of parent object {\tt scattResComp2d}.

Before describing these objects, there are two more things
to address.
First, since the boundary conditions
will be imposed at a certain radius, polar coordinates
are the most convenient way to represent the scattering
problem~(\ref{eqtn-scattprob}). In particular, it will
be useful to have at hand the Laplacian in polar coordinates:
\begin{equation}\label{eqtn-polarlaplacian}
 \Delta = \frac{\partial^2}{\partial r^2} +
          \frac{1}{r} \frac{\partial}{\partial r} +
          \frac{1}{r^2} \frac{\partial^2}{\partial\theta^2}.
\end{equation}
Second, in all cases the discretization of~(\ref{eqtn-scattprob})
will gives us a matrix equation
\begin{equation}\label{eqtn-defT}
 T(k) \widehat{\scatt} = -\widehat{V\inc}
\end{equation}
where the hat indicates evaluation on the user-specified mesh.
The general form of $T(k)$ is $A - k^2 B + C(k)$, where
$A - k^2 B$ encapsulates the PDE and linear parts of
boundary conditions, and $C(k)$ holds the nonlinear parts of
the boundary conditions (e.g. the DtN map). Therefore each
of the child objects will have a method called {\tt T}.
