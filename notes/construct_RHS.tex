\section{Constructing the right-hand side for scattering}
\label{sec-RHS}

To verify this is all conceived of and implemented correctly,
I want to check that scattered waves are indeed outgoing outside
the potential. So, I need to know how to pose the scattering
problem in the discretization.

First of all, the scattering problem is
$(-\Delta + V - k^2)\scatt = -V\inc$, $\scatt$
outgoing. Since each block row of the matrices derived above
corresponds to an equation for a given $\theta_m$, each block
row of the discretized $-V\inc$ should be the evaluation
on a ray corresponding to a $\theta_m$. That is,
\[
 \text{RHS} =
 \begin{bmatrix}
  -V\inc \text{ on mesh along ray }\theta = \theta_1 \\
  \vdots \\
  -V\inc \text{ on mesh along ray }\theta = \theta_{N_t}
 \end{bmatrix}.
\]
In order for the boundary conditions to
be enforced, some rows of RHS will be set to zero.

An incident wave we've used before since it was convenient was
\[
 e^{ikx} = \sum_n i^n J_n(kr) e^{in\theta} \quad (x = r\cos\theta).
\]
The Jacobi-Anger expansion above can be derived 
by getting the Fourier coefficients of $e^{ikx}$
in integral form and then comparing to integral expressions for
first-kind Bessel functions.
