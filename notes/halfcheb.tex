\section{``Half'' Chebyshev radial grid on $B(0,1)$ and derivative matrices}
\label{sec-halfcheb}

Ultimately we want to put a mesh on the unit disk in order
to solve the scattering problem using spectral collocation.
Since I'm told and I intuitively believe that having a cluster
of mesh points near the center of the domain would be overkill,
we don't want to put a full Chebyshev mesh on $[0,1]$. Instead,
we can put down half of one. That's easy, but we also have to
derive matrices which act like $\partial/\partial r$ and
$\partial^2/\partial r^2$ at those mesh points.

In the same way this is discussed in Trefethen's Spectral Methods
in Matlab, let's first think about the situation where
our $r,\theta$ grid consists
of a Chebyshev mesh of $[-1,1]$ in the $r$ direction and an equally
spaced mesh of $2N$ points in the $\theta$ direction, where
$\theta_j = (j-1)\pi/N$, $j = 1, 2, ..., 2N$. 
Let the Chebyshev mesh in the $r$ direction
be split into two equal pieces $[\bar{r}; \hat{r}]$ where
$\hat{r} = -\flipud{\bar{r}}$ is the part of the mesh on $[0,1]$
ordered from smallest to largest. Let $D_r$ and $D_{rr}$ be the
usual Chebyshev matrices for mapping values to derivatives at
the mesh points of $[-1,1]$.

Just as we split up the Chebyshev mesh into $[\bar{r}; \hat{r}]$,
let's also split $D_r$ and $D_{rr}$:
\begin{align*}
 D_r = \left[
       \begin{array}{c|c} E_1 & E_2 \\ \hline
                          E_3 & E_4
       \end{array}\right]
 \qquad
 D_{rr} = \left[
       \begin{array}{c|c} D_1 & D_2 \\ \hline
                          D_3 & D_4
       \end{array}\right].
\end{align*}
Then
\begin{align*}
 \begin{bmatrix}
  \psi_r(\bar{r},\theta_1) \\
  \psi_r(\hat{r},\theta_1) \\
    \vdots                 \\
  \psi_r(\bar{r},\theta_N) \\
  \psi_r(\hat{r},\theta_N) \\
  \psi_r(\bar{r},\theta_{N+1}) \\
  \psi_r(\hat{r},\theta_{N+1}) \\
    \vdots                 \\
  \psi_r(\bar{r},\theta_{2N}) \\
  \psi_r(\hat{r},\theta_{2N})
 \end{bmatrix}
 =
 \operatorname{diag}(D_r, ... , D_r)
 \begin{bmatrix}
  \psi(\bar{r},\theta_1) \\
  \psi(\hat{r},\theta_1) \\
    \vdots                 \\
  \psi(\bar{r},\theta_N) \\
  \psi(\hat{r},\theta_N) \\
  \psi(\bar{r},\theta_{N+1}) \\
  \psi(\hat{r},\theta_{N+1}) \\
    \vdots                 \\
  \psi(\bar{r},\theta_{2N}) \\
  \psi(\hat{r},\theta_{2N})
 \end{bmatrix}.
\end{align*}
Now, with the observation that $\psi(r,\theta) =                                                
\psi(-r,(\theta + \pi)\mod 2\pi)$, and since
$\theta_{j+N} = (j+N)\pi/N = \theta_j + \pi$,
we see
\begin{align*}
 \psi(\bar{r},\theta_j)
  = \psi(-\flipud{\hat{r}},\theta_j)
  = \psi(\flipud{\hat{r}},\theta_{(j+N)\mod 2N})
  = \flipud{\psi(\hat{r},\theta_{(j+N)\mod 2N}) }.
\end{align*}
Therefore, if we know what $\psi(\hat{r},\theta_j)$ is
for all $j = 1,...,N,...,2N$, then we know what all the
$\psi(\bar{r},\theta_j)$ are as well. Hence, the former
are all we need to know, and we've got
\begin{align*}
 \begin{bmatrix}
  \psi_r(\hat{r},\theta_1) \\
   \vdots \\
  \psi_r(\hat{r},\theta_N) \\
  \psi_r(\hat{r},\theta_{N+1}) \\
   \vdots \\
  \psi_r(\hat{r},\theta_{2N})
 \end{bmatrix}
 =
 \begin{bmatrix}
  E_3 & E_4 &        &     &     &     &     &        &     &     \\
      &     & \ddots &     &     &     &     &        &     &     \\
      &     &        & E_3 & E_4 &     &     &        &     &     \\
      &     &        &     &     & E_3 & E_4 &        &     &     \\
      &     &        &     &     &     &     & \ddots &     &     \\
      &     &        &     &     &     &     &        & E_3 & E_4
 \end{bmatrix}
 \begin{bmatrix}
  \psi(\bar{r},\theta_1) \\
  \psi(\hat{r},\theta_1) \\
    \vdots                 \\
  \psi(\bar{r},\theta_N) \\
  \psi(\hat{r},\theta_N) \\
  \psi(\bar{r},\theta_{N+1}) \\
  \psi(\hat{r},\theta_{N+1}) \\
    \vdots                 \\
  \psi(\bar{r},\theta_{2N}) \\
  \psi(\hat{r},\theta_{2N})
 \end{bmatrix}.
\end{align*}
Since $E_3 \psi(\bar{r},\theta_j) = E_3\flipud{\psi(\hat{r},                                    
\theta_{(j+N)\mod 2N})} = \fliplr{E_3}\psi(\hat{r},                                             
\theta_{(j+N)\mod 2N})$, this means
\begin{align*}
 \begin{bmatrix}
  \psi_r(\hat{r},\theta_1) \\
   \vdots \\
  \psi_r(\hat{r},\theta_N) \\
  \psi_r(\hat{r},\theta_{N+1}) \\
   \vdots \\
  \psi_r(\hat{r},\theta_{2N})
 \end{bmatrix}
 &=
 \begin{bmatrix}
  \tE & E_4 &        &     &     &     &     &        &     &     \\
      &     & \ddots &     &     &     &     &        &     &     \\
      &     &        & \tE & E_4 &     &     &        &     &     \\
      &     &        &     &     & \tE & E_4 &        &     &     \\
      &     &        &     &     &     &     & \ddots &     &     \\
      &     &        &     &     &     &     &        & \tE & E_4
 \end{bmatrix}
 \begin{bmatrix}
  \psi(\hat{r},\theta_{N+1}) \\
  \psi(\hat{r},\theta_1) \\
    \vdots                 \\
  \psi(\hat{r},\theta_{2N}) \\
  \psi(\hat{r},\theta_N) \\
  \psi(\hat{r},\theta_1) \\
  \psi(\hat{r},\theta_{N+1}) \\
    \vdots                 \\
  \psi(\hat{r},\theta_N) \\
  \psi(\hat{r},\theta_{2N})
 \end{bmatrix} \\
 &=
 \underbrace{
 \begin{bmatrix}
  E_4 &        &     & \tE &        & \\
      & \ddots &     &     & \ddots & \\
      &        & E_4 &     &        & \tE \\
  \tE &        &     & E_4 &        & \\
      & \ddots &     &     & \ddots & \\
      &        & \tE &     &        & E_4
 \end{bmatrix}
 }_{I_{2N} \otimes E_4 +                                                                        
    \begin{bmatrix} 0 & I_N \\ I_N & 0\end{bmatrix}                                             
    \otimes \tE}
 \begin{bmatrix}
  \psi(\hat{r},\theta_1) \\
    \vdots                 \\
  \psi(\hat{r},\theta_N) \\
  \psi(\hat{r},\theta_{N+1}) \\
    \vdots                 \\
  \psi(\hat{r},\theta_{2N}) \\
 \end{bmatrix}
\end{align*}
where $\fliplr{E_3} = \tE$.
Similarly,
\begin{align*}
 \begin{bmatrix}
  \psi_{rr}(\hat{r},\theta_1) \\ \vdots \\
  \psi_{rr}(\hat{r},\theta_{2N})
 \end{bmatrix}
 =
 \left( I_{2N} \otimes D_4 +
 \begin{bmatrix} 0 & I_N \\ I_N & 0\end{bmatrix}
 \otimes \tilde{D}_3 \right)
 \begin{bmatrix}
  \psi(\hat{r},\theta_1) \\ \vdots \\ \psi(\hat{r},\theta_{2N})
 \end{bmatrix}.
\end{align*}
Therefore, letting $R = \operatorname{diag}(1./\hat{r})$,
\begin{align*}
 &\begin{bmatrix}
  \psi_{rr}(\hat{r},\theta_1   ) + R \psi_r(\hat{r},\theta_1   ) \\
   \vdots \\
  \psi_{rr}(\hat{r},\theta_{2N}) + R \psi_r(\hat{r},\theta_{2N})
 \end{bmatrix} \\
 &=
 \left[
  I_{2N} \otimes D_4 +
 \begin{bmatrix} 0 & I_N \\ I_N & 0\end{bmatrix}
 \otimes \tilde{D}_3
 +
 (I_{2N} \otimes R)
 \left(I_{2N} \otimes E_4 +
    \begin{bmatrix} 0 & I_N \\ I_N & 0\end{bmatrix}
    \otimes \tE\right)
 \right]
 \begin{bmatrix}
  \psi(\hat{r},\theta_1) \\ \vdots \\ \psi(\hat{r},\theta_{2N})
 \end{bmatrix} \\
 &=
 \left[
  I_{2N} \otimes \left( D_4 + R E_4 \right) +
  \begin{bmatrix} 0 & I_N \\ I_N & 0\end{bmatrix} \otimes
  \left( \tilde{D}_3 + R \tE \right)
 \right]
 \begin{bmatrix}
  \psi(\hat{r},\theta_1) \\ \vdots \\ \psi(\hat{r},\theta_{2N})
 \end{bmatrix}.
\end{align*}

