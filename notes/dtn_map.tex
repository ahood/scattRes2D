\section{The DtN map and its derivatives}
\label{sec-dtn_map}

The DtN map $f(k,R)$ is a pseudodifferential
operator, so its action on a function $g(\theta)$ is
computed by three steps: a) write down Fourier series
of $g$, i.e. $\sum_{n} c_n e^{in\theta}$, b) multiply
componentwise, i.e. $c_n \mapsto f_n(k,R)c_n$
(where $f_n(k,R)$ is to be defined), and finally c)
sum to get $\sum_n f_n(k,R) c_n e^{in\theta}$.

The set of Hankel functions
$\lbrace H_n^{(1)}(kr), H_n^{(2)}(kr) \rbrace_{n \in \mathbb{Z}}$
is a basis for the kernel of $-\Delta-k^2$. 
[Side note: $H_n^{(1)} = J_n + iY_n$ and
$H_n^{(2)} = J_n - iY_n$.] Now, the first-kind Hankel
functions $H_n^{(1)}(kr)$ correspond to outgoing waves.
So, in the case where the potential is zero outside the
disk of radius $R$, as we assume, to enforce that a 2D function
$u(r,\theta)$ is outgoing
at the boundary of that disk we need to impose that
the coefficients $c_n$ of the Fourier expansion of
$g(\theta) = u(R,\theta)$ all look like their respective
$H_n^{(1)}(kR)$ counterparts. With a way to write down
values and derivatives of $u$ at the mesh points, the
DtN lets us enforce ``outgoing'' in terms of the
relationship between them. In particular, we can actually
write $u(r,\theta) = \sum_n c_n(r) e^{in\theta}$ ($c_n$
is a function now), and enforce that $c_n(r) = H_n^{(1)}(kr)$
for $r \ge R$ by
$c_n'(R) = k \frac{\left(H_n^{(1)}\right)'(kR)}{H_n^{(1)}(kR)}       
c_n(R)$. Let $f_n(k)$ be that ratio multiplying $c_n(R)$. It will
be referred to as the $n$-th component of the DtN map.

Due to $2H_n' = H_{n-1} - H_{n+1}$, we can write (dropping the
superscript (1))
\[
 f_n(k,R) = k \frac{H_n'(kR)}{H_n(kR)}
          = k \frac{H_{n-1}(kR)-H_{n+1}(kR)}{2H_n(kR)}.
\]

As for the derivative of $f_n(k,R)$ with respect to $k$, we get
\begin{align*}
 f_n'(k,R) &= \frac{H_{n-1}(kR) - H_{n+1}(kR)}{2H_n(kR)}\\ &+
            k\frac{\Big\lbrace H_{n-1}'(kR)R-H_{n+1}'(kR)R \Big\rbrace 2H_n(kR) -
                   \Big\lbrace H_{n-1}(kR) -H_{n+1}(kR) \Big\rbrace 2H_n'(kR)R}
                  {4H_n(kR)^2} \\
         &= \frac{H_{n-1}(kR) - H_{n+1}(kR)}{2H_n(kR)}\\ &+
            kR\frac{H_{n-1}'(kR) - H_{n+1}'(kR)}{2H_n(kR)} -
            kR\frac{H_{n-1}(kR)  - H_{n+1}(kR) }{2H_n(kR)}
             \frac{H_n'(kR)}{H_n(kR)}.
\end{align*}
The first term is equal to $f_n(k,R)/k$, the
middle term is equal to
\[
 kR\frac{H_{n-1}(kR)}{2H_n(kR)}\frac{H_{n-1}'(kR)}{H_{n-1}(kR)} -
 kR\frac{H_{n+1}(kR)}{2H_n(kR)}\frac{H_{n+1}'(kR)}{H_{n+1}(kR)}
 =
 R\frac{H_{n-1}(kR)}{2H_n(kR)}f_{n-1}(k,R) -
 R\frac{H_{n+1}(kR)}{2H_n(kR)}f_{n+1}(k,R),
\]
and the third term is equal to $Rf_n(k,R)^2/k$,
so altogether
\[
 f_n'(k) = \frac{f_n(k,R)}{k}\left(1 - Rf_n(k,R)\right) +
           R\frac{H_{n-1}(kR)f_{n-1}(k,R) - H_{n+1}(kR)f_{n+1}(k,R)}
                 {2H_n(kR)}.
\]
