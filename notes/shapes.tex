\section{The {\tt ellipse} and {\tt rect} shape objects}
\label{sec-ellipse}

First, the user has to specify a region in the complex plane
where accurate
resonance approximations are desired. In theory the
region could be any shape, but for simplicity the user
will be required to specify an ellipse by creating
an {\tt ellipse} instance (call it {\tt ell}). Another handy
shape for storing meshes and plotting is a rectangular 
subset of the complex plane,
implemented as the {\tt rect} object. We deal with this 
one first.

\subsection{Creating an instance of {\tt rect}}

The user passes the $x$-coordinates {\tt x1, x2} 
of the left and right
edges and the $y$-coordinates {\tt y1, y2} of the
bottom and top, and optionally the numbers of mesh points 
{\tt nx, ny} in
the $x$ and $y$ directions. A call looks like
\begin{equation}
 {\tt r = rect(x1,x2,y1,y2,nx,ny);}
\end{equation}
If {\tt nx, ny} not passed, a mesh of the rectangular
region will be created later if needed.

\subsection{{\tt rect} properties and methods}

\begin{description}
 \item[Properties]
   \begin{itemize}
    \item[]
    \item {\tt x1,x2,y1,y2,nx,ny} as described above
    \item {\tt X,Y} are the outputs of meshgrid used for
          making 3d plots and contour plots
    \item {\tt Z} is {\tt X + 1i*Y}
   \end{itemize}
 \item[Methods]
   \begin{itemize}
    \item[]
    \item {\tt set\_mesh(obj,nx,ny)} sets the properties
          {\tt nx,ny,X,Y,Z}
    \item {\tt contains(obj,pts)} returns a boolean vector
          indicating which points in the vector {\tt pts}
          are inside the rectangle
    \item {\tt focus(obj,buff)} moves the viewing window to
          the rectangle plus a buffer--the buffer defaults to
          0 if none is passed
    \item {\tt log10contour(obj,f)} plots the contours of
          $\log_{10}(f(z))$
    \item {\tt draw(obj)} draws the rectangle
   \end{itemize}
\end{description}

\subsection{Creating an instance of {\tt ellipse}}

The ellipse is thought of as a parametrized curve
\begin{equation}
 \varphi(t) = c + e^{i\theta}\left(a\cos(2\pi t) + ib\sin(2\pi t)\right),
 \qquad
 [0,1)
\end{equation}
in the complex plane, where
\begin{itemize}
 \item $a$ and $b$ are its two radii,
 \item $\theta$ is a rotation angle, and
 \item $c$ is the position of its center.
\end{itemize}
To create an object representing this ellipse, do
\begin{equation}
 {\tt ell = ellipse(c,theta,a,b,[],[]);}
\end{equation}
At the time of creation the user can also specify a mesh of this region 
with {\tt nx, ny} equal to the number of mesh points in the
$x$ and $y$ directions:
\begin{equation}
 {\tt ell = ellipse(c,theta,a,b,nx,ny);}
\end{equation}

\subsection{{\tt ellipse} properties and methods}

The object stores the anatomy of the ellipse plus some convenience
functions.

\begin{description}
 \item[Properties]
   \begin{itemize}
    \item[]
    \item {\tt c, theta, a, b} as described above
    \item {\tt f1,f2} the foci
    \item {\tt R} is the sum of the distances from any
          point on the ellipse to the two foci
    \item {\tt bb} is a {\tt rect} instance representing 
          a ``bounding box'' for the ellipse, i.e.
          the smallest rectangle that contains the ellipse
    \item {\tt X,Y} holds a mesh of the bounding box,
          created with a {\tt meshgrid()} call
    \item {\tt Z} equals {\tt X + 1i*Y}
   \end{itemize}
 \item[Methods]
   \begin{itemize}
    \item[]
    \item {\tt set\_mesh(obj,nx,ny)} sets up {\tt X,Y,Z}
    \item {\tt contains(obj,pts)} returns a boolean vector saying
          which points in {\tt pts} are in the ellipse
    \item {\tt phi(obj,t)} is the parametrization described above
    \item {\tt dphi(obj,t)} is the parametrization of $\varphi'(t)$
    \item {\tt draw(obj)} draws a picture of the ellipse
    \item {\tt focus(obj,buff)} moves the viewing window to the bounding
          box plus a buffer around the edge--the default in case {\tt buff}
          is not passed is 0
   \end{itemize}
\end{description}

