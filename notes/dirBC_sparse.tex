\subsection{{\tt dirBC\_sparse}}
\label{sec-dirBC_sparse}

Creation is the same as for a {\tt dirBC} instance.
Solving the scattering problem looks the same, too, 
except that the user can pass a preconditioner and
a flag indicating whether GMRES output should be 
suppressed or shown. The default preconditioner
at the moment is the identity.

A run where the default preconditioner and
incident wave are used
and GMRES output is shown looks like:
\begin{verbatim}
 >> k = 2; // wave number
 >> dirs = dirBC_sparse(Nt,Nrs,Vs,coords,Rs);
 >> scattValuesVec = dirs.solve(k,[],[],1);
\end{verbatim}

In addition to inherited properties and methods,
a {\tt dirBC\_sparse} instance has the following.
\begin{description}
 \item[Properties]
   \begin{itemize}
    \item[]
    \item {\tt VvaluesVec} is a vector of potential
          values at the mesh points
    \item {\tt coords} is the coordinates in which
          the problem was specified, and can be
          {\tt 'rect'}, {\tt 'polar'} or {\tt 'complex'}
   \end{itemize}
 \item[Methods]
   \begin{itemize}
    \item[]
    \item {\tt apply\_T(obj,x0,k)} applies the matrix
          $\dirT(k)$ to vector {\tt x0} without forming it
    \item {\tt apply\_B(obj,x0,k)} applies the matrix
          $\dirB$ to vector {\tt x0} without forming it
    \item {\tt apply\_pc(obj,x0,k)} applies the matrix
          $M(k)^{-1}$ to vector {\tt x0},
          where $M(k)$ is a default
          preconditioner (evaluated at $k$) that is
          not explicitly formed
    \item {\tt eig\_comp(obj,n,E0,verbose)} computes
          the {\tt n} eigenvalues of $(\dirA,\dirB)$ closest
          to complex number {\tt E0} and returns 
          them and their eigenvectors as
          {\tt evecs,evals} (optional last argument
          {\tt verbose} should be set to 1 to see
          output of {\tt gmres} )           
   \end{itemize}
\end{description}

