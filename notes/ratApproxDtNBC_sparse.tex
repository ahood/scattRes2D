\section{{\tt ratApproxDtNBC\_sparse}}
\label{sec-ratApproxDtNBC_sparse}

Besides the option to specify inputs to iterative solvers and needing to
pass a {\tt DtNBC\_sparse} instance instead of a {\tt DtNBC} instance, the user
can treat this object just like {\tt ratApproxDtNBC}.

\subsection{Creating an instance}

Since we're after resonance energies $E = k^2$ where the $k$ values are
characterized as eigenvalues of $\dtnT$, we have to choose whether to
make a rational approximation to $\dtnT(\sqrt{E})$ (primary branch of
square root) or to $\dtnT(-\sqrt{E})$ (second branch). The user will pass
1 or 2 for {\tt br} to reflect their choice. The user will also pass
a region where the rational approximation to $\dtnT(\pm\sqrt{E})$
will be optimized, and a {\tt DtNBC\_sparse} instance. Then creating
an instance of {\tt ratApproxDtNBC\_sparse} looks like
\begin{verbatim}
 >> rats = ratApproxDtNBC_sparse(dtns,ell,br);
\end{verbatim}

\subsection{Properties and methods}

\begin{description}
 \item[Properties]
   \begin{itemize}
    \item[]
    \item {\tt br} specifies a branch of the square root function, where
          1 (the default) is the primary branch and 2 is the second branch
    \item {\tt dtns} is a {\tt DtNBC\_sparse} instance, included so 
          all its properties and methods can be ``inherited'' without
          making {\tt ratApproxDtNBC\_sparse} be a child of that class 
    \item {\tt ell} is an {\tt ellipse} instance representing where
          the rational approximation will be good
    \item {\tt mysqrt} is {\tt @sqrt} or {\tt -@sqrt} depending on {\tt br}
    \item {\tt ratf} is a list of {\tt ratApproxWithPoles} instances for
          components of DtN map
    \item {\tt ratf\_lens} is a list of the size of each rational approximation
          block
    \item {\tt ratf\_ends} is a list of the row indices of the end of each
          rational approximation block
    \item {\tt poles} is a cell array of lists of poles 
          within the ellipse {\tt ell} for each DtN map component
    \item {\tt sc} is a Schur complement object that keeps track of how to
          map $\ratT_{\text{full}}(k)$ to $\ratT(k)$
    \item {\tt A22diag} is a vector of diagonal entries of the
          (diagonal) 2,2 block of {\tt A}, made up of the poles of the
          rational approximation
    \item {\tt blk\_precond\_k} is the value of $k$ last used for the block
          diagonal preconditioner for 
    \item {\tt M11\_blks} is a cell array of blocks used for the block diagonal
          preconditioner for $\ratT_{\text{full}}(k)$
    \item {\tt blk\_fourier\_precond\_k} is the value of $k$ last used for
          the Fourier preconditioner for $\ratT_{\text{full}}(k)$
    \item {\tt fourierS\_blks} is a cell array of blocks used for the Fourier
          preconditioner for $\ratT_{\text{full}}(k)$
    \item {\tt sc\_blk\_fourier\_precond\_k} is the value of $k$ last used
          for the Fourier preconditioner for $\ratT(k)$
    \item {\tt sc\_fourierM\_blks} is a cell array of blocks used for the
          Fourier preconditioner for $\ratT(k)$
   \end{itemize}
 \item[Methods]
   \begin{itemize}
    \item[]
    \item {\tt resonances(obj,n,E0,verbose)} computes the eigenvalues of
          the pencil $(\ratA,\ratB)$ and returns the eigenvectors and eigenvalues
          as {\tt resvecs, resvals}, respectively
    \item {\tt solve(obj,k,incfun,precond,verbose)} returns the vector of
          scattered wave values on the mesh, where {\tt incfun}, {\tt precond},
          and {\tt verbose} are optional and defaults will be used if they're
          not passed
    \item {\tt show\_poles(obj,r)} plots the poles of the rational approximation
          and resizes the window to the {\tt rect} instance {\tt r}
    \item {\tt apply\_T(obj,x0,k)} applies $\ratT(k)$ to input vector
          {\tt x0} and returns the result
    \item {\tt apply\_Tfull(obj,x0,k)} applies $\ratT_{\text{full}}(k)$
          to input vector {\tt x0} and returns the result
    \item {\tt apply\_Bfull(obj,x0)} applies $\ratB$ to input vector
          {\tt x0} and returns the result
    \item {\tt apply\_pc\_blkdiag(obj,x0,k)} applies $M(k)^{-1}$ to vector
          {\tt x0} without forming it, where $M(k)$ is the block diagonal
          preconditioner for $\ratT_{\text{full}}(k)$
    \item {\tt apply\_pc\_fourier\_Vav\_blkdiagICBC(obj,x0,k,kind)}
          applies $M(k)^{-1}$ to vector {\tt x0} without forming it,
          where $M(k)$ is the Fourier preconditioner for
          $\ratT_{\text{full}}(k)$
    \item {\tt apply\_scpc\_fourier\_Vav\_blkdiagICBC(obj,x0,k,kind)}
          applies $M(k)$ or $M(k)^{-1}$ (without forming)
          to vector {\tt x0} depending on
          whether {\tt kind} is {\tt 'mult'} or {\tt 'inv'}, where
          $M(k)$ is the Fourier preconditioner for $\ratT(k)$
    \item {\tt apply\_pc(obj,x0,k)} applies $M(k)^{-1}$ to
          {\tt x0} where $M(k)$ is default preconditioner for
          $\ratT(k)$
    \item {\tt apply\_pc\_full(obj,x0,k)} applies $M(k)^{-1}$ to
          {\tt x0} where $M(k)$ is default preconditioner for
          $\ratT_{\text{full}}(k)$
    \item {\tt apply\_fourierA12(obj,x0)}, {\tt apply\_fourierA21(obj,x0)} 
          apply the matrices $\tilde{A}_{12}$, $\tilde{A}_{21}$ from
          Section~\ref{sec-preconditioners}
    \item {\tt get\_poles\_in\_range(n,region,R,br)} returns the poles
          of the rational approximation (i.e. the poles of
          $\ratC(\pm\sqrt{E})$, depending on {\tt br} and {\tt R}) in the region
          represented by {\tt region}
   \end{itemize}
\end{description}

\subsection{Computing resonances}

Same as for {\tt ratApproxDtNBC}, computing the $n$ resonance energies
closest to $E_0$ using an instance {\tt rats} of {\tt ratApproxDtNBC\_sparse}
looks like
\begin{verbatim}
 >> [resvecs,resvals] = rats.resonances(n,E0);
\end{verbatim}
