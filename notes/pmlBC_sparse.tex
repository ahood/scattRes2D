\subsection{{\tt pmlBC\_sparse}}
\label{sec-pmlBC_sparse}

With this object we can also do scattering
computations, albeit only for the
full mesh including PML region. 
The properties and methods (in addition to those inherited) are:
\begin{description}
 \item[Properties]
   \begin{itemize}
    \item[]
    \item {\tt s}, {\tt ss}, {\tt coords}
          and {\tt pieces} are
          all the same as the {\tt pmlBC} properties
    \item {\tt ds}, {\tt d2s} are derivatives with respect
          to original coordinate {\tt r} (not used by the user)
    \item {\tt VvaluesVec} is a vector of potential values
          at the mesh points
   \end{itemize}
 \item[Methods]
   \begin{itemize}
    \item[]
    \item {\tt RHSfromFun(obj,incfun)}, 
          {\tt RHSfromFun\_full(obj,incfun)}
          are the same as the {\tt pmlBC} methods
    \item {\tt apply\_op\_no\_bcs(obj,x0,E0,VvaluesVec)}
          overrides the {\tt scattResComp2d\_sparse} method
          to account for PML coordinate change
    \item {\tt apply\_Tfull(obj,x0,k)} applies the
          matrix $\pmlT_{\text{full}}(k)$ to vector {\tt x0}
          without creating the matrix
    \item {\tt apply\_Bfull(obj,x0)} applies the
          matrix $\pmlB$ to vector {\tt x0}
          without creating the matrix
    \item {\tt apply\_pc\_full(obj,x0,k)} applies
          $M(k)^{-1}$ to vector {\tt x0} without forming it,
          where $M(k)$ is the default preconditioner for
          the matrix $\pmlT_{\text{full}}(k)$
    \item {\tt solve\_full(obj,k,incfun,precond,verbose)}
          returns the vector of scattered wave values on the 
          mesh that includes the PML region,
          i.e. $\widehat{\scatt}^{(\text{ext})}$,
          where the inputs {\tt incfun}, {\tt precond} and
          {\tt verbose} are optional and defaults will be used
          if they're not passed
    \item {\tt eig\_comp(obj,n,E0,verbose)} computes the {\tt n} eigenvalues
          of the pencil $(\pmlA,\pmlB)$ nearest to {\tt E0} and shows the {\tt gmres}
          output if {\tt verbose} is true--the eigenvalues and their eigenvectors
          are returned as the pair {\tt evecs, evals}
   \end{itemize}
\end{description}
To do the
scattering computation using the default incident wave
and preconditioner (the identity for now)
and suppress GMRES output, do
\begin{verbatim}
 >> k = 2; // wave number
 >> pmls = pmlBC_sparse(Nt,Nrs_in,Nr_out,Vs,coords,Rs_in,R_out,l,dl,d2l);
 >> scattValuesVec_ext = pmls.solve_full(k);
\end{verbatim}
