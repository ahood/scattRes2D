\section{Rational approximations}
\label{sec-rat_approx}

In the end, the goal is to find the resonance energies,
approximated as eigenvalues of $\dtnT(\pm\sqrt{E})$
(approximated because we had to pass from the continuous
definition~(\ref{eqtn-resprob}) to a discretization).
Unfortunately, we don't know how to rewrite
$\dtnT(\pm\sqrt{E})$ as
a standard, generalized, polynomial, or rational
eigenvalue problem. Since $\dtnT(\pm\sqrt{E})
= \dtnA - E \dtnB + \dtnC(\pm\sqrt{E})$, the
trouble is with the nonlinear part $\dtnC(\pm\sqrt{E})$.
So, the best thing we know how to do is
find a rational approximation to $\dtnC(\pm\sqrt{E})$,
meaning a nice matrix-valued function 
$\ratT_{\text{full}}(\pm\sqrt{E}) := \ratA - E\ratB$ 
whose Schur complement
$\ratT(\pm\sqrt{E}) := \dtnA - E\dtnB + \ratC(\pm\sqrt{E})$
satisfies $\ratC(\pm\sqrt{E}) \approx \dtnC(\pm\sqrt{E})$
for energies $E$ in a user-specified region of the complex plane
(an ellipse, for simplicity).
In the following sections, the matrices $\ratA$ and
$\ratB$ will be derived.

\subsection{Rational approx from quadrature (analytic function)}

First of all, suppose an ellipse has been chosen and let
$\Gamma$ be its boundary. (Any region with smooth
boundary would work, but we'll stick with an ellipse for
simplicity.) Let $f$ be a function analytic on and inside $\Gamma$.

Then for any $z_0$ in the interior of $\Gamma$, and for 
$f$ analytic in the interior of $\Gamma$, the Cauchy
integral formula is
\[
 f(z_\ast) = \frac{1}{2\pi i}\int_\Gamma
             \frac{f(z)}{z-z_\ast}\,dz.
\]
For $\varphi : [0,1] \rightarrow \Gamma$,
\begin{align*}
 2\pi i f(z_\ast)
 = \int_0^1 \frac{f(\varphi(t))}{\varphi(t)-z_\ast}
   \,\varphi'(t)\,dt
\end{align*}
Given a mesh $\lbrace 0 = t_0,t_1,...,t_N=1 \rbrace$
of $[0,1]$, the trapezoid rule gives
\begin{align*}
 \int_0^1 g(t)\,dt 
 &= \sum_{j=1}^N \int_{t_{j-1}}^{t_j}
    g(t)\,dt 
    \qquad & \left( g = \frac{f\circ\varphi}
                             {\varphi - z_\ast}\,\varphi' \right) \\ 
 &\approx \sum_{j=1}^N 
    \left[g(t_{j-1}) + g(t_j)\right]
    \frac{h_j}{2} 
    \qquad & \left( h_j = t_j - t_{j-1} \right) \\ 
 &= g(t_0)\,\frac{h_1}{2} + 
    \sum_{j=1}^{N-1}
    g(t_j)\,\frac{h_j + h_{j+1}}{2} + 
    g(t_N)\,\frac{h_N}{2}.
\end{align*}
Letting $z_j = \varphi(t_j)$ and 
\begin{align*}
 w_j &= f(z_j)\,\varphi'(t_j)\,\frac{t_{j+1}-t_{j-1}}{4\pi i},
 \quad j = 1,...,N-1, \\ 
 w_0 &= f(z_0)\,\varphi'(t_0)\,\frac{t_1 - t_0}{4\pi i},
 \quad
 w_N = f(z_N)\,\varphi'(t_N)\,\frac{t_N - t_{N-1}}{4\pi i},
\end{align*}
we get
\[
 f(z_\ast) \approx \sum_{j=0}^N \frac{w_j}{z_j - z_\ast}.
\]
Let $z$ and $w$ be column vectors of the nodes and weights, resp.

\subsection{Rational approx from quadrature (meromorphic function)}

If $f$ has poles, then we can get a rational approximation by
punching holes out of the domain. For example, if 
$f(z) = g(z)/(z-p)$ in the interior of $\Gamma$, and if
$\Gamma_p$ is a little circle around $p$, then for any point
$z_\ast$ in the interior of $\Gamma$ but exterior of $\Gamma_p$,
\[
 2\pi i f(z_\ast) = \left( \int_\Gamma - \int_{\Gamma_p} \right)
         \frac{f(z)}{z-z_\ast}\,dz.
\]
Using the same procedure as in the last section to 
get a quadrature rule for the (clockwise) integral around
$\Gamma_p$, put $z^{(p)},w^{(p)}$ as vectors
of the respective
nodes and weights. Then,
\[
 f(z_\ast) \approx 
   \sum_{j=0}^N \frac{w_j}{z_j-z_\ast} + 
   \sum_{j=0}^{N^{(p)}} \frac{w_j^{(p)}}{z_j^{(p)}-z_\ast}.
\]

It's clear that if we append each of $z^{(p)},w^{(p)}$
to the ends of their counterparts $z,w$, then we end up with
a quadrature rule that looks just like the one for analytic 
functions. Let $\hat{z}$ and $\hat{w}$ be the nodes and weights
for the rational approximation to the meromorphic function $f$.

\subsection{Linearizing a rational approximation}

Suppose we have vectors $\hat{z}$ and $\hat{w}$ of
nodes and poles, defined as above. Then,
the Schur complement of
\[
 \left[
 \begin{array}{c | c}
  0 & \hat{w}^T \\ \hline
  -{\bf 1} & \operatorname{diag}(\hat{z}) - z_\ast I
 \end{array}
 \right]
\]
onto the leading 1,1 block is
\[
 0 - \hat{w}^T \left(\diag(\hat{z}) - z_\ast I\right)^{-1}
     (-{\bf 1})
 = 
 \sum_j \frac{\hat{w}_j}{\hat{z}_j - z_\ast},
\]
which is the rational approximation to $f$ at $z_\ast$
previously derived.

\subsection{Creating $\ratA$ and $\ratB$}

Let $\zn{n}$, $\wn{n}$ be the nodes and
weights for a rational approximation to $f_n(\pm\sqrt{E},R)$,
according to the user's choice
of branch, and $k = \pm\sqrt{E}$ accordingly. 
Then the Schur complement 
$-W\left(\diag(Z) - E I\right)^{-1} P$ of
\begin{align*}
 \left[
 \begin{array}{c|c}
    & W \\ \hline
  P & \diag(Z) - E I
 \end{array}
 \right]
\end{align*}
onto the leading block
is a rational approximation to
\begin{align*}
 \left[
 \begin{array}{cccc|c|cccc}
  0 &        &   &               &        &   &        &   & \\
    & \ddots &   &               &        &   &        &   & \\
    &        & 0 &               &        &   &        &   & \\
    &        &   & f_{-N+1}(k,R) &        &   &        &   & \\ \hline
    &        &   &               & \ddots &   &        &   & \\ \hline
    &        &   &               &        & 0 &        &   & \\
    &        &   &               &        &   & \ddots &   & \\ 
    &        &   &               &        &   &        & 0 & \\ 
    &        &   &               &        &   &        &   & f_N(k,R)
 \end{array}
 \right]
 \qquad
 % ouch, using n in two different ways
 (\diag\left(d(k,R)\right) \otimes e_ne_n^T \text{ from Section~\ref{sec-dtnC}})
\end{align*}
where
\begin{align*}
 W =
 \left[
 \begin{array}{c|c}
  0          & \\
  \vdots     & \\ 
  0          & \\ 
  (\wn{0})^T & \\ \hline
             & 0 \\
             & \vdots \\ 
             & 0 \\ 
             & (\wn{1})^T, 
 \end{array}
 \right], 
 \qquad
 Z = 
 \begin{bmatrix}
  \zn{0} \\ \zn{1}
 \end{bmatrix}, \qquad\text{and}\qquad
 P = 
 \left[
 \begin{array}{cccc|cccc}
  0       & \hdots & 0      & -1     &        &        &        \\
  \vdots  &        & \vdots & \vdots &        &        &        \\
  0       & \hdots & 0      & -1     &        &        &        \\ \hline
          &        &        &        & 0      & \hdots & -1     \\ 
          &        &        &        & \vdots &        & \vdots \\ 
          &        &        &        & 0      & \hdots & -1
 \end{array}
 \right]
\end{align*}
for the case $N = 1$.

Therefore
\begin{align*}
 (U \otimes I)
 \left(\diag\left(d(k,R)\right) \otimes e_ne_n^T\right)
 (U^{-1} \otimes I)
 \approx
 (U \otimes I)
 \left(-W\left(\diag(Z) - E I\right)^{-1}P\right)
 (U^{-1} \otimes I).
\end{align*}
Since $\dtnC(k) = -(U\otimes I)
\left(\diag\left(d(k,R)\right) \otimes e_ne_n^T\right)
(U^{-1}\otimes I)$, this means the Schur complement of
\begin{align*}
 \left[
 \begin{array}{c|c}
                       & (U \otimes I)W \\ \hline
  -P(U^{-1} \otimes I) & \diag(Z) - E I
 \end{array}
 \right]
\end{align*}
is a good approximation to $\dtnC(k)$ for $k^2 = E$
in the user-specified ellipse. Therefore the Schur
complement of
\begin{align*}
 \left[
 \begin{array}{c|c}
  \dtnA - E\dtnB       & (U \otimes I)W \\ \hline
  -P(U^{-1} \otimes I) & \diag(Z) - EI
 \end{array}
 \right]
\end{align*}
is 
\begin{align*}
 \ratT(k) := 
 \dtnA - E \dtnB + 
 \underbrace{
 (U \otimes I)
 W\left(\diag(Z) - EI\right)^{-1} P
 (U^{-1} \otimes I)
 }_{\ratC(k)}
\end{align*}
where $\ratC(k) \approx \dtnC(k)$.
So we define
\begin{align*}
 \ratA
 = 
 \left[
 \begin{array}{c|c}
  \dtnA                & (U \otimes I)W \\ \hline
  -P(U^{-1} \otimes I) & \diag(Z) 
 \end{array}
 \right],
 \qquad
 \ratB
 = 
 \left[
 \begin{array}{c|c}
  \dtnB &  \\ \hline
        & I
 \end{array}
 \right].
\end{align*}

The relationship between these variables and the ones used in
Section~\ref{sec-preconditioners} and the code are summarized in the
following table:
\begin{center}
\begin{tabular}{c c c}
 here    & Section~\ref{sec-preconditioners} & code \\ \hline \\ 
  $W$                    & $\tilde{A}_{12}$ & {\tt fourierA12} \\ 
  $(U \otimes I)W$       & $A_{12}$         & {\tt A12} \\ 
  $-P$                   & $\tilde{A}_{21}$ & {\tt fourierA21} \\ 
  $-P(U^{-1} \otimes I)$ & $A_{21}$         & {\tt A21} \\ 
  $Z$                    & $\diag(D_{22})$  & {\tt A22diag}
\end{tabular}
\end{center}

