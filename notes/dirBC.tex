\section{Solving the scattering problem with {\tt dirBC}}
\label{sec-dirBC}

This object allows the user to solve the
scattering~(\ref{eqtn-scattprob}) and
resonance~(\ref{eqtn-resprob}) problems with Dirichlet boundary
conditions.

\subsection{The boundary condition}
\label{sec-dirBC-BC}

A Dirichlet boundary condition (rather than the outgoing
condition) in~(\ref{eqtn-scattprob})
means enforcing that $\scatt(R) = 0$, giving the following
boundary value problem:
\begin{equation}\label{eqtn-dirprob}
\begin{aligned}
 \left(-\Delta + V - k^2\right)\scatt &= -V\inc
 \qquad
 \text{on }B(0,R), \\
 \scatt(R) &= 0.
\end{aligned} 
\end{equation}
The solution to~(\ref{eqtn-dirprob}) is a good approximation to
the solution to~(\ref{eqtn-scattprob}) in case the potential has 
very thick, high walls that very quickly attenuate the scattered wave.

Discretizing~(\ref{eqtn-dirprob}) amounts to taking the matrices $\dirA$
and $\dirB$ that come from discretizing the PDE and
replacing the boundary condition rows of $\dirA$ and $\dirB$ with
rows of the identity and zeros, respectively. 
Then if $\dirT(k) = \dirA - k^2 \dirB$, 
$\dirT(k)\widehat{\scatt} = -\widehat{V\inc}$,
where the hat indicates values on the mesh of $B(0,R)$.
The object
{\tt dirBC} encapsulates the discretization of~(\ref{eqtn-dirprob}).

\subsection{Creating an instance}
\label{sec-dirBC-create}

Create an instance with
\begin{verbatim}
 >> dir = dirBC(Nt,Nrs,Vs,coords,Rs);
\end{verbatim}
where the potential $V$ is specified with cell array {\tt Vs}
(see above) and {\tt coords} equals 'rect', 'polar', or 'complex'
depending on whether $V = V(x,y)$, $V = V(r,\theta)$, or
$V = V(z)$.

\subsection{Properties and methods}
\label{sec-dirBC-properties}

Here are properties and methods, in addition to those
available through
parent object {\tt scattResComp2d}:
\begin{description}
 \item[Properties]
   \begin{itemize}
    \item[]
    \item {\tt Vs, coords} are as described above
    \item {\tt A, B} are the matrices $\dirA$ and $\dirB$ described above;
          the values of the scattered wave on the mesh solve
          $\dirT(k)\widehat{\scatt} = -\widehat{V\inc}$
%%%%%%%%%%%%%%%%%%%%%%%%%%%%%%%%%%%%%%%%%%%%%%%%%%%%%%                        
% Lr probably doesn't need to be a property                                    
%%%%%%%%%%%%%%%%%%%%%%%%%%%%%%%%%%%%%%%%%%%%%%%%%%%%%%%%%%%%                    
    \item {\tt Lr} is the discretization of the radial part of the
          Laplacian, $\frac{\partial^2}{\partial r^2} +     
          \frac{1}{r}\frac{\partial}{\partial r}$, no boundary
          or interface conditions applied
    \item {\tt evecs,evals} are the eigenvalues $E$ and their
          eigenvalues, computed when {\tt eig\_comp()} is called
    \item {\tt A\_orig}, {\tt B\_orig}, {\tt A\_cc}, {\tt B\_cc},
          aren't important right now and
          not all of them need to be properties, probably
   \end{itemize}
 \item[Methods]
   \begin{itemize}
    \item[]
    \item {\tt enforceBC(obj)} adjusts boundary condition rows of 
          $\dirA$ and $\dirB$
          as described above
    \item {\tt T(obj,k)} returns the matrix $\dirT(k)$
%%%%%%%%%%%%%%%%%%%%%%%%%%%%%%%%%%%%%%%%%%%%%%%%%%%%%%%%%%%     
% kinda outdated
%%%%%%%%%%%%%%%%%%%%%%%%%%%%%%%%%%%%%%%%%%%%%%%%%%%%%%%%%%%%                 
    \item {\tt eig\_comp(obj)} computes the eigenvalues of the pencil
          $(\dirA,\dirB)$ to find resonances
   \end{itemize}
\end{description}

\subsection{Solving the scattering problem}
\label{sec-dirBC-solvescattprob}

The boundary value problem~(\ref{eqtn-dirprob})
is specified completely with the following three things:
\begin{itemize}
 \item a potential function $V$ (supported in some $B(0,R)$),
 \item wave number $k$, and
 \item incident wave $\inc$ satisfying $-\Delta\inc = k^2 \inc$.
\end{itemize}
Analogously, 
the discretization of~(\ref{eqtn-dirprob}) is specified completely
with 
\begin{itemize}
 \item a {\tt dirBC} instance, 
 \item a wave number {\tt k}, and 
 \item an incident wave {\tt incfun}. 
\end{itemize}
Given {\tt k} the default incident wave
is {\tt incfun = @(x,y) exp(1i*k*x)}.

Supposing a {\tt dirBC} instance (let's call it {\tt dir}) 
has already been created and the user
wants to use the default incident wave {\tt incfun = @(x,y) exp(1i*k*x)},
solving the scattering problem with frequency parameter
{\tt k} can be done in one line:
\begin{verbatim}
 >> scattValuesVec = dir.solve(k);
\end{verbatim}

If the user wants to use a different incident wave (satisfying
$-\Delta\inc = k^2\inc$), solving is still a one-liner:
\begin{verbatim}
 >> scattValuesVec = dir.solve(k,incfun);
\end{verbatim}

Here is 
a full example using the default $\inc = e^{ikx}$ and a piece-wise
constant ring potential equal to 0 on $0 \le r \le 1$ and
equal to 1 on $1 < r \le 2$:
\begin{verbatim}
 >> Vs = {@(r,t) 0*r; @(r,t) 1 + 0*r};
 >> coords = 'polar';
 >> Nt = 20;
 >> Nrs = [30, 30];
 >> Rs = [1,2];
 >> dir = dirBC(Nt,Nrs,Vs,coords,Rs);
 >> k = 2;
 >> scattValuesVec = dir.solve(k);
\end{verbatim}
