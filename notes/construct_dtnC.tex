\section{Constructing $\dtnC(k)$}
\label{sec-dtnC}

The boundary condition is
$\psi_r(R,\theta) = f(k)\psi(R,\theta)$ for all $\theta$, which
we will enforce at our $\theta_j$'s.
As shown in Section~\ref{sec-piecewise},
it's easy to map our vector of $\psi$ values to $\psi_r$ values.
But in order to apply the DtN map $f(k)$, we have to go to
Fourier space.

\subsection{Fourier coefficients to values with $U$}

The first step is to find the mapping of Fourier coefficients to values.
The Fourier expansion 
$\psi(R,\theta) = \sum_{n=-\infty}^\infty \psi_n(R)e^{in\theta}$
is an infinite sum, so for the purposes of discretization 
it must be truncated. Since there are
$2N$ $\theta$ mesh points, we can keep $2N$ Fourier coefficients
$\psi_n(R)$. Due to the truncation, the following is only
approximate, but an equals sign will be used for simplicity because that's
how the discretization is implemented. The mapping from Fourier coefficients
to values is then
\begin{align*}
 \begin{bmatrix}
  \psi(R,\theta_1) \\ \psi(R,\theta_2) \\ \vdots \\ \psi(R,\theta_{2N})
 \end{bmatrix}
 =
 \underbrace{
 \begin{bmatrix}
  e^{i(-N+1)\theta_1}    & e^{i(-N+2)\theta_1}    & \hdots & e^{iN\theta_1} \\
  e^{i(-N+1)\theta_2}    & e^{i(-N+2)\theta_2}    & \hdots & e^{iN\theta_2} \\ 
        \vdots           & \vdots                 & \ddots &      \vdots    \\
  e^{i(-N+1)\theta_{2N}} & e^{i(-N+2)\theta_{2N}} & \hdots & e^{iN\theta_{2N}}
 \end{bmatrix}
 }_U
 \begin{bmatrix}
  \psi_{-N + 1}(R) \\ \psi_{-N+2}(R) \\ \vdots \\ \psi_{N}(R)
 \end{bmatrix}.
\end{align*}

Keeping in mind that we'd like to be acting on vectors of values at
all mesh points, what we really need is a map from the Fourier coefficients
evaluated on the radial mesh, i.e.
\begin{align*}
 \begin{bmatrix}
  \psi_{-N+1}(\hat{r}) \\ \psi_{-N+2}(\hat{r}) \\ \vdots \\ \psi_N(\hat{r})
 \end{bmatrix}
 \mapsto
 \begin{bmatrix}
  \psi(\hat{r},\theta_1) \\ \psi(\hat{r},\theta_2) \\ \vdots \\ 
  \psi(\hat{r},\theta_{2N})
 \end{bmatrix}.
\end{align*}
Since $U$ maps the Fourier coefficients at $r$ to values at $r$ for 
every $r$ and not just $R$, 
\begin{align*}
 \begin{bmatrix}
  \psi(\hat{r},\theta_1) \\ \psi(\hat{r},\theta_2) \\ \vdots \\ 
  \psi(\hat{r},\theta_{2N})
 \end{bmatrix}.
 = 
 \underbrace{
 \begin{bmatrix}
  e^{i(-N+1)\theta_1}I    & e^{i(-N+2)\theta_1}I    & \hdots & e^{iN\theta_1}I \\
  e^{i(-N+1)\theta_2}I    & e^{i(-N+2)\theta_2}I    & \hdots & e^{iN\theta_2}I \\ 
        \vdots            & \vdots                  & \ddots &      \vdots     \\
  e^{i(-N+1)\theta_{2N}}I & e^{i(-N+2)\theta_{2N}}I & \hdots & e^{iN\theta_{2N}}I
 \end{bmatrix}
 }_{U \otimes I}
 \begin{bmatrix}
  \psi_{-N+1}(\hat{r}) \\ \psi_{-N+2}(\hat{r}) \\ \vdots \\ \psi_N(\hat{r})
 \end{bmatrix}
\end{align*}


\subsection{Applying $U$ with the inverse DFT}

Being a map from fourier coefficients to values, it's not suprising
that this is almost an inverse discrete fourier transform. From
Wikipedia 
(\url{https://en.wikipedia.org/wiki/Discrete_Fourier_transform#Definition}),
the inverse DFT maps $\lbrace X_0, X_1, ..., X_{N-1} \rbrace$
to $\lbrace x_0, x_1, ..., x_{N-1} \rbrace$ under
$x_n = \frac{1}{N} \sum_{k=0}^{N-1} X_k e^{2\pi i k n/N}$. In terms of
our indexing and notation (starting indices from 1 and using $2N$ points
instead of $N$, the DFT maps $\lbrace X_1,X_2,...,X_{2N} \rbrace$
to $\lbrace x_1,x_2,...,x_{2N} \rbrace$ under
$x_n = \frac{1}{2N} \sum_{k=1}^{2N} X_k e^{\pi i (k-1)(n-1)/N}$.
Noticing that $\theta_n = (n-1)\pi/N$, the map is
\begin{align*}
x_n = \frac{1}{2N} \sum_{k=1}^{2N} X_k e^{i(k-1) \theta_n}.
\end{align*}

The matrix equation for the DFT in our notation is therefore
\begin{align*}
 \begin{bmatrix}
  x_1 \\ x_2 \\ \vdots \\ x_{2N}
 \end{bmatrix}
 =
 \frac{1}{2N} 
 \begin{bmatrix}
  1       & 1                & \hdots & 1 \\ 
  1       & e^{i\theta_2}    & \hdots & e^{i(2N-1)\theta_2} \\ 
   \vdots & \vdots           & \ddots &  \vdots \\ 
  1       & e^{i\theta_{2N}} & \hdots & e^{i(2N-1)\theta_{2N}}
 \end{bmatrix}
 \begin{bmatrix}
  X_1 \\ X_2 \\ \vdots \\ X_{2N}
 \end{bmatrix}
\end{align*}
Keeping in mind that $\theta_1 = 0$, it's now clear that 
a diagonal scaling turns that matrix into $U$:
\begin{align*}
 U =
 \begin{bmatrix}
  e^{i(-N+1)\theta_1} &                     &        & \\ 
                      & e^{i(-N+1)\theta_2} &        & \\ 
                      &                     & \ddots & \\ 
                      &                     &        & e^{i(-N+1)\theta_{2N}}.
 \end{bmatrix}
 \begin{bmatrix}
  1       & 1                & \hdots & 1 \\ 
  1       & e^{i\theta_2}    & \hdots & e^{i(2N-1)\theta_2} \\ 
   \vdots & \vdots           & \ddots &  \vdots \\ 
  1       & e^{i\theta_{2N}} & \hdots & e^{i(2N-1)\theta_{2N}}
 \end{bmatrix}.
\end{align*}
So, if $\hat{X}$ is a vector, then
\begin{align*}
 U\hat{X} 
 = 2N
  \begin{bmatrix}
  e^{i(-N+1)\theta_1} &                     &        & \\ 
                      & e^{i(-N+1)\theta_2} &        & \\ 
                      &                     & \ddots & \\ 
                      &                     &        & e^{i(-N+1)\theta_{2N}}.
 \end{bmatrix}
 {\tt ifft}(\hat{X})
\end{align*}
where {\tt ifft} is the Matlab function.

\subsection{The DtN map in Fourier space}

In Fourier space, the DtN map is a diagonal operator, so
in our discretization the mapping on the Fourier coefficients
of $\psi(R,\theta)$ is 
\begin{align*}
 \begin{bmatrix}
  \psi_{-N+1}(R) \\ \psi_{-N+2}(R) \\ \vdots \\ \psi_N(R)
 \end{bmatrix}
 \mapsto
 \underbrace{
 \begin{bmatrix}
  f_{-N+1}(k,R) &               &        & \\
                & f_{-N+2}(k,R) &        & \\ 
                &               & \ddots & \\ 
                &               &        & f_N(k,R)
 \end{bmatrix}
 }_{\diag\left(d(k,R)\right)}
 \begin{bmatrix}
  \psi_{-N+1}(R) \\ \psi_{-N+2}(R) \\ \vdots \\ \psi_N(R)
 \end{bmatrix}.
\end{align*}
Therefore, the action on the vector of Fourier coefficients evaluated
on the radial mesh is
\begin{align*}
 \begin{bmatrix}
  \psi_{-N+1}(\hat{r}) \\ \hline
  \vdots \\ \hline
  \psi_N(\hat{r})
 \end{bmatrix}
 \mapsto
 \underbrace{
 \left[
 \begin{array}{cccc|c|cccc}
  0 &        &   &               &        &   &        &   & \\ 
    & \ddots &   &               &        &   &        &   & \\ 
    &        & 0 &               &        &   &        &   & \\ 
    &        &   & f_{-N+1}(k,R) &        &   &        &   & \\ \hline
    &        &   &               & \ddots &   &        &   & \\ \hline
    &        &   &               &        & 0 &        &   & \\ 
    &        &   &               &        &   & \ddots &   & \\ 
    &        &   &               &        &   &        & 0 & \\ 
    &        &   &               &        &   &        &   & f_N(k,R)
 \end{array}
 \right]
 }_{\diag\left(d(k,R)\right) \otimes e_ne_n^T}
 \begin{bmatrix}
  \psi_{-N+1}(\hat{r}) \\ \hline
  \vdots \\ \hline
  \psi_N(\hat{r})
 \end{bmatrix}
\end{align*}

\subsection{The DtN map boundary condition and $\dtnC(k)$}

Using the work from the previous sections, in our
discretization the DtN map $f(k,R)$ acts like
\begin{align*}
 f(k,R)
 \begin{bmatrix}
  \psi(\hat{r},\theta_1) \\ \vdots \\ \psi(\hat{r},\theta_{2N})
 \end{bmatrix}
 = 
 \underbrace{
 (U \otimes I)
 \left(\diag\left(d(k,R)\right) \otimes e_ne_n^T\right)(U \otimes I)^{-1}
 }_{U\diag\left(d(k,R)\right)U^{-1} \otimes e_ne_n^T}
 \begin{bmatrix}
  \psi(\hat{r},\theta_1) \\ \vdots \\ \psi(\hat{r},\theta_{2N})
 \end{bmatrix}
\end{align*}

Since the DtN map boundary condition is 
\begin{align*}
 \left(\partial/\partial r - f(k,R)\right)\psi = 0,
\end{align*}
enforcing this in the discretization involves updating
the last row in each block of $A$ so they map to
$\psi_r(R,\theta_j)$ for each $\theta_j$ and setting
\begin{align*}
 \dtnC(k) = -U\diag\left(d(k,R)\right)U^{-1} \otimes e_ne_n^T.
\end{align*}

