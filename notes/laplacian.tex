\section{Discretized Laplacian}
\label{sec-laplacian}

We want to discover (complex) frequencies which are
poles of the Schr\"odinger operator, i.e. resonances.
The time-independent Schr\"odinger equation is
$(-\Delta + V - k^2)\psi = 0$, where $V$ is a potential
function which we'll assume to be piecewise smooth (for
the sake of making our discretization scheme manageable)
and we write the resonance energy $E$ as $k^2$ 
for notational convenience. If $\psi$
is outgoing, then $k^2$ is a resonance energy. In this section,
boundary conditions will be ignored.

We want to enforce that our proposed solution $\psi$ satisfies
$(-\Delta +V - k^2)\psi = 0$ at an $r,\theta$ mesh of a disc.
We'll use an equally spaced mesh of $2N$ points
in the $\theta$ direction (the requirement for evenness
emerges in Section~\ref{sec-halfcheb}).
Those $\theta$ grid points will be $0, \Delta\theta, 2\Delta\theta, ...
, (2N-1)\Delta\theta$ where $\Delta\theta = 2\pi/(2N)$. 
The $r$ direction will be meshed as described in
Section~\ref{sec-piecewise}. Let the vector $\hat{r}$ be the
vector of the $n$ mesh points.

Now, in polar coordinates $\Delta = \frac{\partial^2}{\partial r^2}                         
+ \frac{1}{r} \frac{\partial}{\partial r} + \frac{1}{r^2}                    
\frac{\partial^2}{\partial \theta^2}$. 
The radial derivatives part of the Laplacian has already been
discretized in Section~\ref{sec-piecewise}.
So, we only need to worry about discretizing the $\theta$ derivative
term. 


Suppose we have a matrix $D_{\theta\theta}$ which acts as
\begin{align*}
 D_{\theta\theta}\psi(r,\hat{\theta})
  = \psi_{\theta\theta}(r,\hat{\theta}).
\end{align*}
Then,
\begin{align*}
 \underbrace{
 \begin{bmatrix}
  D_{\theta\theta} &        & \\
                   & \ddots & \\
                   &        & D_{\theta\theta}
 \end{bmatrix}
 }_{I \otimes D_{\theta\theta}}
 \begin{bmatrix}
  \psi(r_1,\hat{\theta}) \\ \vdots \\ \psi(r_n,\hat{\theta})
 \end{bmatrix}
 =
 \begin{bmatrix}
  \psi_{\theta\theta}(r_1,\hat{\theta}) \\
  \vdots \\
  \psi_{\theta\theta}(r_n,\hat{\theta})
 \end{bmatrix}
\end{align*}

There is a permutation matrix $P$ such that
\begin{align*}
 P
 \begin{bmatrix}
  \psi(\hat{r},\theta_1)\\ \vdots \\ \psi(\hat{r},\theta_N)
 \end{bmatrix}
 =
 \begin{bmatrix}
  \psi(r_1,\hat{\theta}) \\
  \vdots \\
  \psi(r_n,\hat{\theta})
 \end{bmatrix}
\end{align*}
and hence
\begin{align*}
 P^T (I \otimes D_{\theta\theta}) P
 \begin{bmatrix}
  \psi(\hat{r},\theta_1) \\ \vdots \\ \psi(\hat{r},\theta_N)
 \end{bmatrix}
 =
 \begin{bmatrix}
  \psi_{\theta\theta}(\hat{r},\theta_1) \\
  \vdots \\
  \psi_{\theta\theta}(\hat{r},\theta_N)
 \end{bmatrix}
\end{align*}

It turns out that $P^T (I \otimes D_{\theta\theta}) P
 = D_{\theta\theta} \otimes I$, shown as follows.

\begin{align*}
 \left(\Dtt \otimes I\right)
 \begin{bmatrix} 
  \psi(\hat{r},\theta_1) \\ 
  \vdots \\ 
  \psi(\hat{r},\theta_N)
 \end{bmatrix}
 &= \vec\left(
     I
     \begin{bmatrix}
      \psi(\hat{r},\theta_1) \hdots \psi(\hat{r},\theta_N)
     \end{bmatrix}
     \Dtt^T \right) \\ 
 &= \vec\left(
     \begin{bmatrix}
      \psi(r_1,\hat{\theta})
      \hdots
      \psi(r_n,\hat{\theta})
     \end{bmatrix}^T \Dtt^T \right) \\ 
 &= \vec\left(\left(
     \Dtt
     \begin{bmatrix}
      \psi(r_1,\hat{\theta})
      \hdots
      \psi(r_n,\hat{\theta})
     \end{bmatrix}
    \right)^T \right) \\ 
 &= \vec\left(
     \begin{bmatrix}
      \psi_{\theta\theta}(r_1,\hat{\theta})
      \hdots
      \psi_{\theta\theta}(r_n,\hat{\theta})
     \end{bmatrix}^T \right) \\ 
 &= \vec\left(
     \begin{bmatrix}
      \psi_{\theta\theta}(\hat{r},\theta_1)
      \hdots
      \psi_{\theta\theta}(\hat{r},\theta_N)
     \end{bmatrix} \right) \\ 
 &= \begin{bmatrix}
     \psi_{\theta\theta}(\hat{r},\theta_1) \\ 
     \vdots \\ 
     \psi_{\theta\theta}(\hat{r},\theta_N)
    \end{bmatrix}
\end{align*}

For $R$ the diagonal matrix of the reciprocals of
the mesh values ({\tt 1./}$\hat{r}$ in semi-Matlab notation), 
\begin{align*}
 \begin{bmatrix}
  \Delta\psi(\hat{r},\theta_1) \\ \vdots \\
  \Delta\psi(\hat{r},\theta_N)
 \end{bmatrix}
 &=
 \begin{bmatrix}
  \psi_{rr}(\hat{r},\theta_1) +
  R\psi_r(\hat{r},\theta_1) +
  R^2\psi_{\theta\theta}(\hat{r},\theta_1) \\ \vdots \\
  \psi_{rr}(\hat{r},\theta_N) +
  R\psi_r(\hat{r},\theta_N) +
  R^2\psi_{\theta\theta}(\hat{r},\theta_N)
 \end{bmatrix} \\
 &=
 \left( {\tt Drr} + (I \otimes R) {\tt Dr}
                  + (I \otimes R^2) (\Dtt \otimes I)
 \right)
 \begin{bmatrix}
  \psi(\hat{r},\theta_1) \\ 
  \vdots \\ 
  \psi(\hat{r},\theta_N)
 \end{bmatrix}
\end{align*}
where
\begin{align*}
 {\tt Drr} &= I_{2N} \otimes {\tt Drr\_I} + 
              \begin{bmatrix}0 & I_N \\ I_N & 0\end{bmatrix}
              \otimes {\tt Drr\_J} \\ 
 {\tt Dr}  &= I_{2N} \otimes {\tt Dr\_I} + 
              \begin{bmatrix}0 & I_N \\ I_N & 0\end{bmatrix}
              \otimes {\tt Dr\_J},
\end{align*}
for {\tt Drr\_I}, {\tt Drr\_J}, {\tt Dr\_I} and {\tt Dr\_J}
as defined in Section~\ref{sec-piecewise}.

Due to the property that $(A \otimes B)(C \otimes D) = 
AC \otimes BD$, our discrete Laplace operator is
therefore
\begin{align*}
 \hat{\Delta} = {\tt Drr} + (I \otimes R){\tt Dr} + \Dtt \otimes R^2.
\end{align*}

Finally, enforcing $(-\Delta + V - k^2)\psi = 0$ at all mesh points
gives us
\begin{align*}
 \left(
  -\hat{\Delta} 
  +\operatorname{diag}(V(\hat{r},\theta_1),...,V(\hat{r},\theta_N))
  -k^2 I 
 \right)
 \begin{bmatrix}
  \psi(\hat{r},\theta_1) \\ \vdots \\ \psi(\hat{r},\theta_N)
 \end{bmatrix} = 0,
\end{align*}
where the zero on the right-hand side is interpreted to be the values
of the zero function on our mesh.
