\section{Piecewise discretizations in $r$ direction}
\label{sec-piecewise}

This is all well and good, but we might want to discretize
intervals $[0,R_1]$, $[R_1,R_2]$, etc. in the $r$ direction separately
(specifically in the case where $V$ is axisymmetric but only
piecewise continuous). To be specific, suppose we want to mesh $[0,R_1]$ and
$[R_1,R_2]$ separately with $\hat{r}_1$ and $\hat{r}_2$ (note that
the last component of $\hat{r}_1$ will be equal to the first
component of $\hat{r}_2$, and $\hat{r}_2$ will be a full Chebyshev
mesh). If we want to discretize how the Laplacian acts on
$[\hat{r}_1, \hat{r}_2]$, we have to rework things a bit. Let's start
with the discretization of $\partial/\partial r$.

Using our work above and letting $D_r$ be the Chebyshev differentiation
matrix for the interval $[R_1,R_2]$,
we have
\begin{align*}
 \left[
 \begin{array}{c|c}
  I_{2N} \otimes E_4 +
  \begin{bmatrix}
   0   & I_N \\
   I_N & 0
  \end{bmatrix} \otimes \tilde{E}_3 &    \\ \hline
  &
  \begin{array}{cccc}
   D_r &     &        & \\
       & D_r &        & \\
       &     & \ddots & \\
       &     &        & D_r
  \end{array}
 \end{array}
 \right]
 \begin{bmatrix}
  \psi(\hat{r}_1,\theta_1) \\ \vdots \\ \psi(\hat{r}_1,\theta_{2N}) \\ \hline
  \psi(\hat{r}_2,\theta_1) \\ \vdots \\ \psi(\hat{r}_2,\theta_{2N})
 \end{bmatrix}
 =
 \begin{bmatrix}
  \psi_r(\hat{r}_1,\theta_1) \\ \vdots \\ \psi_r(\hat{r}_1,\theta_{2N}) \\ \hline
  \psi_r(\hat{r}_2,\theta_1) \\ \vdots \\ \psi_r(\hat{r}_2,\theta_{2N})
 \end{bmatrix}.
\end{align*}
For the case $N = 2$, this is
\begin{align*}
 \left[
  \begin{array}{c|c}
   \begin{array}{c|c}
    \begin{array}{cc} E_4 &  \\   & E_4 \end{array}
    &
    \begin{array}{cc} \tilde{E}_3 & \\ & \tilde{E}_3 \end{array} \\ \hline
    \begin{array}{cc} \tilde{E}_3 & \\ & \tilde{E}_3 \end{array}
    &
    \begin{array}{cc} E_4 & \\ & E_4 \end{array}
   \end{array}
  & \\ \hline
  &
  \begin{array}{cccc}
   D_r &     &     & \\
       & D_r &     & \\
       &     & D_r & \\
       &     &     & D_r
  \end{array}
 \end{array}
 \right]
 \begin{bmatrix}
  \psi(\hat{r}_1,\theta_1) \\
  \psi(\hat{r}_1,\theta_2) \\
  \psi(\hat{r}_1,\theta_3) \\
  \psi(\hat{r}_1,\theta_4) \\ \hline
  \psi(\hat{r}_2,\theta_1) \\
  \psi(\hat{r}_2,\theta_2) \\
  \psi(\hat{r}_2,\theta_3) \\
  \psi(\hat{r}_2,\theta_4)
 \end{bmatrix}
 =
 \begin{bmatrix}
  \psi_r(\hat{r}_1,\theta_1) \\
  \psi_r(\hat{r}_1,\theta_2) \\
  \psi_r(\hat{r}_1,\theta_3) \\
  \psi_r(\hat{r}_1,\theta_4) \\ \hline
  \psi_r(\hat{r}_2,\theta_1) \\
  \psi_r(\hat{r}_2,\theta_2) \\
  \psi_r(\hat{r}_2,\theta_3) \\
  \psi_r(\hat{r}_2,\theta_4)
 \end{bmatrix}
\end{align*}
which we can rewrite as
\begin{align*}
 \left[
 \begin{array}{c|c}
  \begin{array}{cccc}
   E_4 &     &     & \\
       & D_r &     & \\
       &     & E_4 & \\
       &     &     & D_r
  \end{array}
  &
  \begin{array}{cccc}
   \tilde{E}_3 & \quad &             & \quad \\
               & \quad &             & \quad \\
               & \quad & \tilde{E}_3 & \quad \\
               & \quad &             & \quad
  \end{array}
  \\ \hline
  \begin{array}{cccc}
   \tilde{E}_3 & \quad &             & \quad \\
               & \quad &             & \quad \\
               & \quad & \tilde{E}_3 & \quad \\
               & \quad &             & \quad
  \end{array}
  &
  \begin{array}{cccc}
   E_4 &     &     & \\
       & D_r &     & \\
       &     & E_4 & \\
       &     &     & D_r
  \end{array}
 \end{array}
 \right]
 \begin{bmatrix}
  \psi(\hat{r}_1,\theta_1) \\
  \psi(\hat{r}_2,\theta_1) \\
  \psi(\hat{r}_1,\theta_2) \\
  \psi(\hat{r}_2,\theta_2) \\ \hline
  \psi(\hat{r}_1,\theta_3) \\
  \psi(\hat{r}_2,\theta_3) \\
  \psi(\hat{r}_1,\theta_4) \\
  \psi(\hat{r}_2,\theta_4)
 \end{bmatrix}
 =
 \begin{bmatrix}
  \psi_r(\hat{r}_1,\theta_1) \\
  \psi_r(\hat{r}_2,\theta_1) \\
  \psi_r(\hat{r}_1,\theta_2) \\
  \psi_r(\hat{r}_2,\theta_2) \\ \hline
  \psi_r(\hat{r}_1,\theta_3) \\
  \psi_r(\hat{r}_2,\theta_3) \\
  \psi_r(\hat{r}_1,\theta_4) \\
  \psi_r(\hat{r}_2,\theta_4)
 \end{bmatrix}.
\end{align*}
Now the pattern can be seen, and so if $\hat{r}$ is the concatenation
of $n$ piecewise discretizations, the first being a half Chebyshev
mesh and the others being Chebyshev meshes with derivative matrices
$D_r^{(j)}$, $j = 2,3,...n$, we get
\begin{align*}
 \left(
  I_{2N} \otimes
  \underbrace{
  \begin{bmatrix}
   E_4 &           &        & \\
       & D_r^{(2)} &        & \\
       &           & \ddots & \\
       &           &        & D_r^{(n)}
  \end{bmatrix}
  }_{\tt Dr\_I}
   +
  \begin{bmatrix} 0 & I_N \\ I_N & 0 \end{bmatrix} \otimes
  \underbrace{
  \begin{bmatrix}
   \tilde{E}_3 &   &        & \\
               & 0 &        & \\
               &   & \ddots & \\
               &   &        & 0
  \end{bmatrix}
  }_{\tt Dr\_J}
 \right)
 \begin{bmatrix}
  \psi(\hat{r},\theta_1) \\ \vdots \\ \psi(\hat{r},\theta_{2N})
 \end{bmatrix}
 =
 \begin{bmatrix}
  \psi_r(\hat{r},\theta_1) \\ \vdots \\ \psi_r(\hat{r},\theta_{2N})
 \end{bmatrix}.
\end{align*}
Similarly,
\begin{align*}
 \left(
  I_{2N} \otimes
  \underbrace{
  \begin{bmatrix}
   D_4 &              &        & \\
       & D_{rr}^{(2)} &        & \\
       &              & \ddots & \\
       &              &        & D_{rr}^{(n)}
  \end{bmatrix} 
  }_{\tt Drr\_I}
  +
  \begin{bmatrix} 0 & I_N \\ I_N & 0 \end{bmatrix} \otimes
  \underbrace{
  \begin{bmatrix}
   \tilde{D}_3 &   &        & \\
               & 0 &        & \\
               &   & \ddots & \\
               &   &        & 0
  \end{bmatrix}
  }_{\tt Drr\_J}
 \right)
 \begin{bmatrix}
  \psi(\hat{r},\theta_1) \\ \vdots \\ \psi(\hat{r},\theta_{2N})
 \end{bmatrix}
 =
 \begin{bmatrix}
  \psi_{rr}(\hat{r},\theta_1) \\ \vdots \\ \psi_{rr}(\hat{r},\theta_{2N})
 \end{bmatrix}.
\end{align*}
