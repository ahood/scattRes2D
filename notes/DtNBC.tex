\section{Solving the scattering problem with {\tt DtNBC}}
\label{sec-DtNBC}

A DtN map boundary condition is an exact boundary condition
for the scattering problem, meaning that an outgoing wave
(and only an outgoing wave) will satisfy this boundary condition
exactly. Therefore, out of the three methods we present here,
we expect this one to produce the most accurate solution
to~(\ref{eqtn-scattprob}).

\subsection{The boundary condition}
\label{sec-DtNBC-BC}

The natural setting for the DtN map is in Fourier space.
Specifically, 
\begin{equation}\label{eqtn-dtnmap}
 \sum_n c_n e^{in\theta} 
 \quad\xmapsto{\text{DtN map}}\quad
 \sum_n f_n(k,R) c_n e^{in\theta},
 \qquad
 f_n(k,R) = k\frac{\left(H_n^{(1)}\right)'(kR)}{H_n^{(1)}(kR)}.
\end{equation}

The set of first- and second-kind Hankel functions 
$\lbrace H_n^{(1)}(kr), H_n^{(2)}(kr) \rbrace_{n=-\infty}^\infty$
form a basis for the solution space of the PDE
in~(\ref{eqtn-scattprob}). 
The first-kind Hankel functions $H_n^{(1)}(kr)$ are a basis for the
space of outgoing waves. 
So, a general solution 
$\sum_n c_n(r) e^{in\theta}$ with
$c_n(r) = a_n H_n^{(1)}(kr) + b_n H_n^{(2)}(kr)$
is an outgoing wave only if all the $b_n$'s are zero.
To guarantee this, we require
\begin{equation}\label{eqtn-dtnbc}
 c_n'(R) = f_n(k,R) c_n(R) \qquad \forall n,
\end{equation}
which is what we mean by the {\bf DtN map boundary condition}.
See Section~\ref{sec-dtn_map} for a derivation.

Now, we can't enforce this for all $n$. The best we can do is
to enforce it for $n = -{\tt Nt}/2 + 1, ..., 0, 1, 2, ... {\tt Nt}/2$,
that is, a number of Fourier coefficients equal to the number of
$\theta$ mesh points and corresponding to lowest frequencies.
To apply the DtN map to the vector
$x = [c_{-{\tt Nt}/2 + 1}({\tt r}), ..., c_{{\tt Nt}/2}({\tt r})]$
of these Fourier coefficients evaluated on the radial
mesh {\tt r}, we have to multiply by the diagonal matrix
$X(k) = \diag\left([f_{-{\tt Nt}/2 + 1}(k,R), ... , f_{{\tt Nt}/2}(k,R)]\right)
\otimes e_{{\tt Nr}} e_{{\tt Nr}}^T$. Therefore, applying
the DtN map to $\widehat{\scatt}$ involves transforming to
the Fourier representation $x$, applying the diagonal operator, and
transforming back. The matrix accomplishing this is
\begin{equation}
 \dtnC(k) = -({\tt U} \otimes I) X(k) ({\tt Uinv} \otimes I),
\end{equation}
with {\tt U, Uinv} as described in Section~\ref{sec-scattResComp2d-fourier}
and full derivation of $\dtnC(k)$ in Section~\ref{sec-dtnC}.
The discretization of~(\ref{eqtn-scattprob}) with DtN map
boundary conditions will then be
\begin{equation}\label{eqtn-Tdtn}
 \left( \dtnA - k^2 \dtnB + \dtnC(k) \right) \widehat{\scatt} = -\widehat{V\inc}.
\end{equation}
Comparison with~(\ref{eqtn-defT}) shows that we should take
$\dtnT(k) = \dtnA - k^2 \dtnB + \dtnC(k)$.

\subsection{Creating an instance}
\label{sec-DtNBC-create}

Same inputs as {\tt dirBC}:
\begin{verbatim}
 >> dtn = DtNBC(Nt,Nrs,Vs,coords,Rs);
\end{verbatim}

\subsection{Properties and methods}
\label{sec-DtNBC-properties}

Inherits properties and methods of parent {\tt scattResComp2d}, and
here's more:
\begin{description}
 \item[Properties]
   \begin{itemize}
    \item[]
    \item {\tt A,B} are the matrices $\dtnA$ and $\dtnB$
    \item {\tt Vs,coords} are the same as for {\tt dirBC}
    \item {\tt Lr} is matrix of radial part of discretization of Laplacian
    \item {\tt A\_orig, B\_orig, A\_cc, B\_cc} not important right now 
   \end{itemize}
 \item[Methods]
   \begin{itemize}
    \item[]
    \item {\tt enforceBC(obj)} sets the boundary condition 
          rows of $\dtnA$ with derivative part of boundary condition
          and zeros those rows of $\dtnB$
    \item {\tt enforceBlockBC(obj,n,X)}
          is called by {\tt enforceBC()}
    \item {\tt enforceBC\_cc(obj)} isn't important right now
    \item {\tt T(obj,k)} returns $\dtnT(k)$ matrix described above
    \item {\tt dT(obj,k), d2T(obj,k)} return the first and second
          derivatives of $\dtnT(k)$ with respect to $k$
    \item {\tt T\_cc(obj,k)} not important right now
    \item {\tt C(obj,k)} returns the matrix $\dtnC(k)$ described above
    \item {\tt C\_cc(obj,k)} not important right now
    \item {\tt dLogTz(obj,z,br), d2LogTz(obj,z,br)} not important right now
   \end{itemize}
 \item[Static Methods]
   \begin{itemize}
    \item[]
    \item {\tt DtNcoeffs(n,k,R)} returns $f_n(k,R)$ as described above
    \item {\tt dDtNcoeffs(n,k,R)} returns first derivative of $f_n$ with
          respect to $k$, evaluated at $k,R$
    \item {\tt d2DtNcoeffs(n,k,R)} returns second derivative with respect
          to $k$, evaluated at $k,R$
   \end{itemize}    
\end{description}

\subsection{Solving the scattering problem}
\label{sec-DtNBC-solvescattprob}

See Section~\ref{sec-dirBC-solvescattprob} for all details, since 
this works the same way for both {\tt dirBC} and {\tt DtNBC}.
Here is
a full example using the default $\inc = e^{ikx}$ and a piece-wise
constant ring potential equal to 0 on $0 \le r \le 1$ and
equal to 1 on $1 < r \le 2$:
\begin{verbatim}
 >> Vs = {@(r,t) 0*r; @(r,t) 1 + 0*r};
 >> coords = 'polar';
 >> Nt = 20;
 >> Nrs = [30, 30];
 >> Rs = [1,2];
 >> dtn = DtNBC(Nt,Nrs,Vs,coords,Rs);
 >> k = 2;
 >> scattValuesVec = dtn.solve(k);
\end{verbatim}

