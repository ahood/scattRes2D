\section{Using {\tt scattResComp2d}}
\label{sec-scattResComp2d}

The {\tt scattResComp2d} object does the heavy lifting of
bookkeeping, discretizing, and plotting, as well as
having a few other conveniences.

Creating an instance is simple: 
\begin{verbatim}
 >> s = scattResComp2d(Nt,Nrs,Rs);
\end{verbatim}
where
\begin{itemize}
 \item {\tt Nt} is the number of mesh points in the $\theta$
                 direction ({\bf must be even} because of the way
                 the Laplacian is constructed--if you pass an odd
                 {\tt Nt}, it will be increased by one)
 \item {\tt Nrs} is a vector holding the number of mesh points in
                 the radial direction on each concentric subregion
                 (see Sections~\ref{sec-mesh} and~\ref{sec-potential})
 \item {\tt Rs} is a vector holding the outer radius of each
                 subregion.
\end{itemize}
What follows is a description of all properties and methods, grouped
by usage.

\subsection{Easy indexing}
\label{sec-scattResComp2d-indexing}

\begin{description}
 \item[Properties]
   \begin{itemize}
    \item[] % so bullet starts on next line
    \item {\tt Nr} is the total number of mesh points in the 
                 radial direction (sum over {\tt Nrs})
    \item {\tt BCrows} is the vector of row indices where the boundary
                 conditions will be placed
    \item {\tt valRows} is the vector of row indices in each block row
                 where continuity of the scattered wave 
                 across subregions is enforced
    \item {\tt derRows} same as {\tt valRows}, but for enforcing
                 differentiability
    \item {\tt ICBCrows} is the (ordered) vector of row indices where
                 continuity, differentiability, and boundary
                 conditions are enforced
   \end{itemize}
\end{description}

\subsection{Easy mesh representations}
\label{sec-scattResComp2d-mesh}

\begin{description}
 \item[Properties]
   \begin{itemize}  
    \item[] % so bullet starts on next line
    \item {\tt r} is the vector of mesh points in the radial direction
    \item {\tt theta} is the vector of mesh points in the theta direction,
          starting from 0 and not including $2\pi$.
    \item {\tt rr,tt} are matrices of mesh point locations in 
          polar coordinates (from {\tt meshgrid}), used for 3d plotting
    \item {\tt xx,yy} are matrices of mesh point locations in rectangular
          coordinates, same ordering as {\tt rr,tt}
    \item {\tt rs,ts} are vectors of mesh point locations in
          polar coordinates, where {\tt rs} is a stack of copies of the
          vector {\tt r},
          and {\tt ts} has {\tt Nr} copies of {\tt theta(1)} at the top,
          followed by {\tt Nr} copies of {\tt theta(2)}, etc.
    \item {\tt xs,ys} are vectors of mesh point locations in
          rectangular coordinates (same order as {\tt rs,ts})
    \item {\tt zs} is a vector of mesh point locations represented
          in complex coordinates ($z = x + iy$)
   \end{itemize}

 \item[Methods]
   \begin{itemize}
    \item[] % so bullet starts on next line
    \item {\tt valuesVecFromFun(obj,f,coords)} 
          takes a function handle {\tt f} representing a 
          function $f$ on the 
          plane and returns its values on the mesh of $B(0,R)$:
         \begin{description}
          \item[{\tt valuesVecFromFun(obj,f,'rect')}] implies
          $f = f(x,y)$, and returns 
          {\tt f(xs,ys)}, 
          \item[{\tt valuesVecFromFun(obj,f,'polar')}] implies
          $f = f(r,\theta)$ and returns
          {\tt f(rs,ts)}, and
          \item[{\tt valuesVecFromFun(obj,f,'complex')}] implies
          $f = f(z)$ and returns
          {\tt f(zs)}
         \end{description}
    \item {\tt valuesVecFromFunCellArray(obj,fs,coords)} is same
          as {\tt valuesVecFromFun} except {\tt fs} is
          a cell array of functions which represent a piecewise
          continuous function on $B(0,R)$ by continuous functions
          defined for each subregion separately
    \item {\tt plotValuesVec(obj,valuesVec,titlestr,part)}
          makes a 3d plot of the specified part 
          (usually real, imag, or abs)
          of a function given its 
          valuesVec representation, and uses {\tt titlestr} 
          as the title of the
          figure ({\tt part} defaults to {\tt @real} 
          if {\tt []} is passed)
    \item {\tt plotFun(obj,f,titlestr,part,coords)} makes
          3d plot of the specified part (usually real, imag,
          or abs) of the function f
          ({\tt part} defaults to {\tt @real} if {\tt []} is passed.
   \end{itemize}
\end{description}

\subsection{Fourier representations and converting}
\label{sec-scattResComp2d-fourier}

\begin{description}
 \item[Properties]
   \begin{itemize}
    \item[] % so bullet starts on next line
    \item {\tt Ns} is the vector of indices of Fourier series terms that
                 would be computed with the discrete Fourier transform
                 ({\tt -Nt/2 + 1:Nt/2})
    \item {\tt U} is the matrix that maps Fourier coefficients 
                  (with indices {\tt Ns}) to function values on
                  {\tt theta}
    \item {\tt Uinv} is the inverse of {\tt U}, mapping function values
                  to Fourier coefficients
   \end{itemize}

 \item[Methods]
   \begin{itemize}
    \item[] % so bullet starts on next line
    \item {\tt fourierVecFromValuesVec(obj,valuesVec)} 
          transforms a valuesVec into a stack of Fourier
          coefficents with indices {\tt Ns}, each evaluated
          on the radial mesh {\tt r}
    \item {\tt valuesVecFromFourierVec(obj,fourierVec)}
          does the opposite and allows for more or
          fewer Fourier coefficients
    \item {\tt fourierVecFromFun(obj,f,coords)} finds the valuesVec
          for {\tt f} and transforms it to a fourierVec
    \item {\tt fourierVecFromFunCellArray(obj,fs,coords)}
          does what {\tt valuesVecFromFunCellArray} does then
          transforms to a fourierVec
    \item {\tt plotFourierVec(obj,fourierVec,titlestr,part)}
          turns a fourierVec into a valuesVec and plots using
          {\tt plotValuesVec}
    \item {\tt chebVecFromFourierVec(obj,fourierVec)} takes each
          $f_n(\hat{r})$ (where {\tt fourierVec} is a stack of these),
          applies {\tt radialVals2chebCoeffs} (see below)
          and returns the stack of these results
    \item {\tt fourierVecFromChebVec(obj,chebVec)} does the opposite,
          applying {\tt chebCoeffs2radialVals}          
    \item {\tt radialVals2chebCoeffs(obj,v,parity)} takes the values
          {\tt v} of an even or odd function evaluated on the radial mesh
          {\tt r} ({\tt parity} is {\tt 'even'} or {\tt 'odd'}), splits
          into $v_1$, $v_2$, ... according to piecewise radial mesh sizes,
          finds the Chebyshev coefficients $c_m$ of each $v_m$
          (more detail below), and returns the stack of $c_m$'s
    \item {\tt chebCoeffs2radialVals(obj,c,parity)} reverses the process
    \item {\tt halfChebVals2chebCoeffs(v,parity)} takes a vector {\tt v}
          of values of even or odd function $f$ ({\tt parity} is
          {\tt 'even'} or {\tt 'odd'}) on a half Chebyshev mesh of $[0,1]$
          and as many Chebyshev coefficients of $f$ as there are
          values in {\tt v}
    \item {\tt chebCoeffs2halfChebVals(c,parity)} reverses the process
    \item {\tt fullChebVals2chebCoeffs(v)} takes a vector {\tt v} of
          values of a function $f$ on a Chebyshev mesh of $[0,1]$ and
          returns as many Chebyshev coefficients as there are values
    \item {\tt chebCoeffs2fullChebVals(c)} reverses the process
   \end{itemize}
\end{description}

\subsection{Chebyshev representations and converting}
\label{sec-scattResComp2d-chebyshev}

\begin{description}
 \item[Properties]
   \begin{itemize}
    \item[] % so bullet starts on next line
    \item {\tt Winv\_even} is a matrix such that {\tt Winv\_even*c}
          equals {\tt chebCoeffs2radialVals(obj,c,'even')} (see below)
    \item {\tt Winv\_odd} is a matrix such that {\tt Winv\_odd*c}
          equals {\tt chebCoeffs2radialVals(obj,c,'odd')} (see below)
   \end{itemize}
 \item[Methods]
   \begin{itemize}
    \item[] % so bullet starts on next line
    \item {\tt chebVecFromFourierVec(obj,fourierVec)} takes each
          $f_n(\hat{r})$ (where {\tt fourierVec} is a stack of these),
          applies {\tt radialVals2chebCoeffs} (see below)
          and returns the stack of these results
    \item {\tt fourierVecFromChebVec(obj,chebVec)} does the opposite,
          applying {\tt chebCoeffs2radialVals}          
    \item {\tt chebVecFromValuesVec(obj,valuesVec)} converts the stack
          of function values to a stack of Fourier coefficient values to
          a stack of Chebyshev coefficients for those piecewise-defined
          Fourier coefficient functions
    \item {\tt valuesVecFromChebVec(obj,chebVec)} does the opposite
    \item {\tt radialVals2chebCoeffs(obj,v,parity)} takes the values
          {\tt v} of an even or odd function evaluated on the radial mesh
          {\tt r} ({\tt parity} is {\tt 'even'} or {\tt 'odd'}), splits
          into $v_1$, $v_2$, ... according to piecewise radial mesh sizes,
          finds the Chebyshev coefficients $c_m$ of each $v_m$
          (more detail below), and returns the stack of $c_m$'s
    \item {\tt chebCoeffs2radialVals(obj,c,parity)} reverses the process
    \item {\tt halfChebVals2chebCoeffs(v,parity)} takes a vector {\tt v}
          of values of even or odd function $f$ ({\tt parity} is
          {\tt 'even'} or {\tt 'odd'}) on a half Chebyshev mesh of $[0,1]$
          and as many coefficients of the expansion of $f$ in even or odd
          Chebyshev polynomials as there are values in {\tt v} (see
          Section~\ref{sec-cheb_basis})
    \item {\tt chebCoeffs2halfChebVals(c,parity)} reverses the process
    \item {\tt fullChebVals2chebCoeffs(v)} takes a vector {\tt v} of
          values of a function $f$ on a Chebyshev mesh of $[0,1]$ and
          returns as many Chebyshev coefficients as there are values
    \item {\tt chebCoeffs2fullChebVals(c)} reverses the process
   \end{itemize}
\end{description}

\subsection{Creating discretized scattering problem}
\label{sec-scattResComp2d-create}

\begin{description}
 \item[Properties]
   \begin{itemize}
    \item[] % so bullet starts on next line
    \item {\tt Dr} is the matrix that maps $f({\tt rs,ts})$ to
          $\frac{\partial f}{\partial r}({\tt rs,ts})$
    \item {\tt Drr} is the matrix that maps $f({\tt rs,ts})$ to
          $\frac{\partial^2 f}{\partial r^2}({\tt rs,ts})$
    \item {\tt Dtt} is the matrix that maps $f({\tt theta})$ to
          $\frac{\partial^2 f}{\partial\theta^2}({\tt theta})$
          (see Section~\ref{sec-laplacian} for how this relates to discretization
          of last term in Laplacian)
    \item {\tt Drs} is not interesting right now
    \item {\tt Dr\_I}, {\tt Dr\_J}, {\tt Drr\_I}, {\tt Drr\_J}
          are matrices used to construct {\tt Dr} and {\tt Drr}
          according to 
          \begin{verbatim}
           Dr  = kron(It,Dr_I)  + kron(Jt,Dr_J)
           Drr = kron(It,Drr_I) + kron(Jt,Drr_J)
          \end{verbatim}
          (see Section~\ref{sec-piecewise} for derivation of
          {\tt Dr\_I, Dr\_J, Drr\_I} and {\tt Drr\_J})
    \item {\tt It}, {\tt Jt} are {\tt Nt} $\times$ {\tt Nt}
          matrices equaling the identity and 
          $\begin{bmatrix} 0 & I_{half} \\ I_{half} & 0\end{bmatrix}$,
          respectively, where $I_{half}$ is the {\tt Nt/2} identity
          ({\bf this is why {\tt Nt} should be even})
   \end{itemize}

 \item[Methods]
   \begin{itemize}
    \item[] % so bullet starts on next line
    \item {\tt enforceDE} is called only by the child objects
          corresponding to boundary condition choice, and constructs
          the matrices $A$ and $B$ such that $A-k^2B$ is the discretization
          of $-\Delta + V - k^2 I$
    \item {\tt enforceIC, enforceBlockIC}
          replaces the {\tt valRows} and {\tt derRows} of $A$ with
          equations enforcing continuity and differentiability 
          at subregion interfaces and zeros the same rows of $B$
    \item {\tt RHSfromFun(obj,incfun)} will only be called
          after a {\tt dirBC}, {\tt pmlBC} or {\tt DtNBC}
          instance is created, and returns
          the valuesVec for $-V\inc$ 
          (the Right Hand Side of~(\ref{eqtn-scattprob}))
          but with the boundary and
          interface condition rows set to zero--note that 
          {\tt incfun} must be in the same
          coordinates which are specified by {\tt coords}
          when the {\tt *BC} instance is created.
          {\bf Alternatively, the user can pass a wave number $k$
          in place of {\tt incfun}
          and the default {\tt incfun = @(x,y) exp(1i*k*x)} will be used.}
    \item {\tt solve(obj,k,incfun)} will only be called after
          a {\tt dirBC}, {\tt pmlBC} or {\tt DtNBC} instance
          is created and returns the values of the approximate scattered
          wave on the user's mesh of $B(0,R)$--the last argument should
          be omitted to use the default $\inc$. For testing purposes,
          if the user passes a vector for {\tt incfun} the
          vector {\tt T(k)\textbackslash incfun} will be returned.
   \end{itemize}
\end{description}

\subsection{Bessel functions and resonance processing}
\label{sec-scattResComp2d-bessel}

\begin{description}
 \item[Static methods]
   \begin{itemize}
    \item[] % so bullet starts on next line
    \item {\tt sort\_eigs} sorts a vector of eigenvalues from 
          left to right in the complex plane and applies same
          ordering to eigenvectors
    \item {\tt get\_closest(p,n,p0)} takes a vector {\tt p} of
          points and returns a vector of the {\tt n} points
          closest to {\tt p0}
    \item {\tt J, Y} are wrappers for the first and second kind
          Bessel functions built in to \textsc{Matlab}, e.g.
          {\tt J(n,k,x) = besselj(n,k*x)}, and 
          {\tt J(n,k,x)} $= J_n(kx)$
    \item {\tt H} is a wrapper for the first kind Hankel function
          built in to \textsc{Matlab} (first kind corresponds to
          outgoing waves)
    \item {\tt dJ, dY, dH} are derivatives of {\tt J, Y, H} 
          with respect to the third variable, e.g.
          {\tt dJ(n,k,x)} $= k J_n'(kx)$
   \end{itemize}
\end{description}

\subsection{Not important right now}
\label{sec-scattResComp2d-unimportant}

\begin{description}
 \item[Properties not worth describing right now]
   \begin{itemize}
    \item[] % so bullet starts on next line
    \item {\tt valRows\_cc}, {\tt derRows\_cc}, 
          {\tt ArowChange}, {\tt BrowChange}, 
          {\tt rowPlacement}
   \end{itemize}

 \item[Methods not worth describing right now]
   \begin{itemize}
    \item[] % so bullet starts on next line
    \item {\tt RHSfromFun\_cc}, {\tt dLogTz},
          {\tt plotRayValuesVec}
   \end{itemize}
\end{description}
