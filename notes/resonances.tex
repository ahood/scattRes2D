\section{Resonances}
\label{sec-resonances}

Recall that given a piecewise smooth potential $V$ supported
in $B(0,R)$, the resonance problem is to find energies
$E = k^2$ such that
\begin{equation}
\begin{aligned}
 \left(-\Delta + V - k^2\right)&\psi = 0 &\qquad\text{on } B(0,R) \\ 
 &\psi \text{ outgoing} &\qquad\text{on } \partial B(0,R)
\end{aligned}
\end{equation}
has a nonzero solution $\psi$. This came from taking the scattering
problem~(\ref{eqtn-scattprob}) and setting $\inc$ equal to zero.
Analogously, the discretized resonance problem is to find values of
$E = k^2$ such that
\begin{equation}
 T(k)\widehat{\psi} = 0
\end{equation}
has a nonzero solution, where $T(k)$ is a matrix-valued
function
(compare with~(\ref{eqtn-defT})).
Finding the values of $k$ such that $T(k)$ is a singular matrix is a
nonlinear eigenvalue problem. As previously mentioned, though, $k$ 
is not the quantity of interest. Instead, we will ultimately want
to rephrase everything in terms of $E = k^2$. Therefore we'll look
for the eigenvalues of $T(\sqrt{E})$ and $T(-\sqrt{E})$ (two problems
corresponding to different branches of the square root).

If Dirichlet boundary conditions are used, we're looking for
eigenvalues of $\dirT(k) = \dirA - k^2 \dirB = \dirA - E \dirB$.
So,
finding resonance approximations is easy--just a generalized 
eigenvalue problem. The trade-off is that the approximations will
usually be lousy, and completely irrelevant ``far''
from the origin. To compute them in the code, just create
a {\tt dirBC} instance (call it {\tt dir}) and do
\begin{verbatim}
 >> dir.eig_comp();
\end{verbatim}
and then access the eigenvalues through the property {\tt dir.evals}.
The {\tt dirBC\_sparse.eig\_comp()} method requires the user to
pass in {\tt E0} and {\tt n} and the $n$ eigenvalues nearest to
$E_0 \in \bbC$ will be returned. If {\tt dirs} is a
{\tt dirBC\_sparse} instance, a call looks like
\begin{verbatim}
 >> n = 10;
 >> E0 = 2;
 >> [evecs,evals] = dirs.eig_comp(E0,n);
\end{verbatim}
To see the {\tt gmres()} output generated during the call, pass {\tt 1}
as an additional argument.

If we use a PML with Dirichlet boundary conditions, things are
a little more complicated, because 
$\pmlT(k) = A_{11} - E B_{11} + \pmlC(k)$
and $\pmlC(k) = -A_{12} \left(A_{22} - E B_{22}\right)^{-1} A_{21}$.
But this is still not too bad--a rational eigenvalue problem. 
A little Gaussian elimination shows that
its eigenvalues are the eigenvalues of 
$T_{\text{full}}(k) = \pmlA - E \pmlB$, so again we have a
generalized eigenvalue problem.

Now, some of the resonance
approximations can be made pretty good by choosing the right PML. 
But no matter what PML is chosen
\begin{itemize} 
 \item accuracy will always be best near the origin and diminish
       as you walk away from it, and
 \item a prerequisite to a larger region of accuracy  
       is a larger mesh, i.e. 
       it takes more computing time and memory to set up and solve
       for resonance approximations that are far from the origin.
\end{itemize}
Furthermore, if the user is interested in resonances in a region
far from the origin, it's not clear how to choose PML parameters
to give the required level of accuracy in that region.
Having said that, computing resonance approximations is 
done the same way as for the {\tt dirBC} instance:
\begin{verbatim}
 >> pml.eig_comp();
\end{verbatim}
where {\tt pml} is an instance of {\tt pmlBC}
set up with the PML of choice. If {\tt pmls} is a {\tt pmlBC\_sparse}
instance, computing eigenvalues looks like
\begin{verbatim}
 >> n = 10; 
 >> E0 = 2;
 >> verbose = 1;
 >> [evecs,evals] = pmls.eig_comp(n,E0,verbose);
\end{verbatim}

The DtN map boundary condition is the only one of these three that 
can exactly enforce that the solution wave is outgoing on
$\partial B(0,R)$. But this boundary condition leads to 
$\dtnT(k) = \dtnA - k^2 \dtnB + \dtnC(k)$, where
$\dtnC(k)$ is a matrix-valued function where entries involve
ratios of Hankel functions and $k$ appears instead of its square. 
This is not a rational eigenvalue problem
like in the case of the PML + Dirichlet boundary condition, so 
it can't be solved exactly. However, if the user specifies a region
of $E$ values in which to look for resonances, we can get
good resonance approximations in that region by approximating
$\dtnC(k)$ by a customizable rational function. As for the
appearance of $k$, not its square,
in $\dtnC(k)$, a branch for $k$ has to be chosen in order to
do a resonance computation.
The {\tt ratApproxDtNBC}
and {\tt ratApproxDtNBC\_sparse} objects will hold the rational
approximation information and the method for resonance computation
given the branch choice.
The following sections will go through the above process step by step,
starting with how to specify a region of interest.

