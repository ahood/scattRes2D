\subsection{{\tt DtNBC\_sparse}}
\label{sec-DtNBC_sparse}

As in the cases of {\tt dirBC\_sparse} and {\tt pmlBC\_sparse}
objects, this one can be treated the same as its non-sparse counterpart.
The major difference is that the user can pass a preconditioner. If none
is passed the default (currently the Fourier preconditioner) will be used. 
See the section on preconditioners for its derivation.
Some of the properties are used to store things for the preconditioners.

In addition to the properties and methods inherited from
{\tt scattResComp2d\_sparse}, there is also
\begin{description}
 \item[Properties]
   \begin{itemize}
    \item[]
    \item {\tt VvaluesVec} is a vector of potential values at the mesh points
    \item {\tt coords} as usual
    \item {\tt Vhat} is a matrix of potential values at the mesh points
    \item {\tt Vav} is a vector of the average potential values on each 
          concentric circle
    \item {\tt A\_blks\_common} is the part of the diagonal blocks of {\tt A}
          that is the same for all blocks
    \item {\tt blk\_precond\_k} is the last value of $k$ used by the 
          block diagonal preconditioner
          (stored so we know if {\tt M\_blks} should be recomputed)
    \item {\tt M\_blks} is a cell array where the $j$th element is a struct
          with fields {\tt L} and {\tt U} storing the LU factorization of
          the $j$th diagonal block of the matrix $\dtnT(k)$
    \item {\tt blk\_fourier\_precond\_k} is the last value of $k$ used
          by the Fourier preconditioner (stored so we know if
          {\tt fourierM\_blks} should be recomputed)
   \end{itemize}
 \item[Methods]
   \begin{itemize}
    \item[]
    \item {\tt apply\_T(obj,x0,k)} applies $\dtnT(k)$ to input vector
          {\tt x0}
    \item {\tt apply\_pc\_blkdiag(obj,x0,k)} applies $M(k)^{-1}$ to
          input vector {\tt x0}, where $M(k)$ is the block diagonal
          preconditioner
    \item {\tt apply\_pc\_fourier\_Vav\_blkdiagICBC(obj,x0,k,kind)}
          takes the fourier preconditioner $M(k)$ and applies either
          $M(k)$ or $M(k)^{-1}$ to input vector {\tt x0}, depending
          on whether {\tt kind} is {\tt 'mult'} or {\tt 'inv'}
    \item {\tt apply\_pc(obj,x0,k)} applies $M(k)^{-1}$ to input
          vector {\tt x0} where $M(k)$ is whichever preconditioner
          has been set as the default (currently the Fourier preconditioner)
    \item {\tt DtNcoeffs(n,k,R)}, {\tt dDtNcoeffs(n,k,R)}, 
          {\tt d2DtNcoeffs(n,k,R)} are same as for {\tt DtNBC} object
   \end{itemize}
\end{description}


Creating an instance (call it {\tt dtns}) 
and doing a scattering computation with the default
preconditioner, the default incident wave, and suppressing
{\tt gmres} output looks like
\begin{verbatim}
 >> k = 2; // wave number
 >> dtns = DtNBC_sparse(Nt,Nrs,Vs,coords,Rs);
 >> scattValuesVec = dtns.solve(k);
\end{verbatim}
For a resonance computation, use {\tt ratApproxDtNBC\_sparse}.




