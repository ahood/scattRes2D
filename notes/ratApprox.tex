\section{{\tt ratApprox} and {\tt ratApproxWithPoles}}
\label{sec-ratApprox}

As derived in Section~\ref{sec-rat_approx}, we can construct
rational approximations to analytic functions. The object
{\tt ratApprox} holds all the relevant pieces and convenience functions.
Rational approximations to meromorphic functions are represented with
{\tt ratApproxWithPoles} objects.

\subsection{Creating a {\tt ratApprox} instance}

The user has to specify a region (elliptical, for simplicity) and
a handle for an analytic function. 
It would make sense for the user to input some
kind of error criterion, but for now they have the choice of giving a
vector holding a mesh of $[0,1]$, including $0$ and $1$. This is optional
and if no mesh is passed the default will be {\tt linspace(0,1,100)}.

Creating an instance of a rational approximation to $\sin(z)$ on
the unit disk could look like
\begin{equation}
\begin{aligned}
 &{\tt >> ell = ellipse(0,0,1,1,[],[]);} \\ 
 &{\tt >> f = @(z) sin(z);} \\ 
 &{\tt >> ratf = ratApprox(ell,f);}
\end{aligned}
\end{equation}

\subsection{{\tt ratApprox} properties and methods}

\begin{description}
 \item[Properties]
   \begin{itemize}
    \item[]
    \item {\tt region} is an {\tt ellipse} instance (or another
          type of object with the same properties and fields, as
          long as it represents a shape with smooth boundary)
    \item {\tt f} is the function handle representing the analytic
          function
    \item {\tt z,w} are vectors of nodes and weights, respectively
   \end{itemize}
 \item[Methods]
   \begin{itemize}
    \item[]
    \item {\tt eval\_at(obj,z0)} returns the value of the rational
          approximation at point {\tt z0} in the complex plane
    \item {\tt compute\_error(obj,r)} computes the matrix of 
          pointwise absolute errors
          in the rational approximation on the mesh of
          some rectangle represented by
          {\tt rect} instance {\tt r}
    \item {\tt show\_error(obj,r)} makes a contour plot of the $\log_{10}$
          absolute error on {\tt rect} instance {\tt r}
   \end{itemize}
\end{description}

\subsection{Child object {\tt ratApproxWithPoles}}

The {\tt ratApproxWithPoles} object is a child of {\tt ratApprox}.
Additional parameters used to create an instance are
\begin{itemize}
 \item vector {\tt poles} contains the locations of poles of
       the meromorphic function {\tt f} inside the ellipse {\tt ell}
 \item vector {\tt rs} of radii of tiny circles around each pole
       (see Section~\ref{sec-rat_approx})    
\end{itemize}
For example, to make a rational approximation to $f(z) = 1/\sin(z)$
on $B(0,1) - B(0,0.01)$ using 50 equally spaced points on
$\partial B(0,1)$, do
\begin{equation}
\begin{aligned}
 &{\tt >> ell = ellipse(0,0,1,1,[],[]);} \\ 
 &{\tt >> f = @(z) 1./sin(z);} \\ 
 &{\tt >> poles = [0];} \\
 &{\tt >> rs = [0.1];} \\ 
 &{\tt >> t = linspace(0,1,50);} \\ 
 &{\tt >> ratf = ratApproxWithPoles(ell,f,poles,rs,t);}
\end{aligned}
\end{equation}

The properties and methods are exactly the same as for the parent object,
but with the addition of {\tt poles} and {\tt rs} properties.
